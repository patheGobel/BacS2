\documentclass[12pt]{article}
\usepackage{stmaryrd}
\usepackage{graphicx}
\usepackage[utf8]{inputenc}

\usepackage[french]{babel}
\usepackage[T1]{fontenc}
\usepackage{hyperref}
\usepackage{verbatim}

\usepackage{color, soul}

\usepackage{pgfplots}
\pgfplotsset{compat=1.15}
\usepackage{mathrsfs}

\usepackage{amsmath}
\usepackage{amsfonts}
\usepackage{amssymb}
\usepackage{tkz-tab}

\usepackage{tikz}
\usetikzlibrary{arrows, shapes.geometric, fit}


\usepackage[margin=2cm]{geometry}

\begin{document}

\begin{minipage}{0.8\textwidth}
	Pathé BA                          
\end{minipage}
\begin{minipage}{0.8\textwidth}
	BAC 2023
\end{minipage}

\begin{center}
\textbf{{\underline{\textcolor{red}{Premier Groupe Correction}}}}
\end{center}
\section*{\textcolor{green}{\underline{Correction Exercice 1} (3 points) :}}
1) Soit \(a\) un nombre rationnel strictement positif et \(n\) un entier naturel. Donner les limites suivantes :

a) 
\[ \lim_{x \to 0} \frac{\ln(x+1)}{x}=1 \]
\[ \textcolor{green}{\underline{\text{Preuve}}} \]

\[
\lim_{x \to 0} \frac{\ln(x+1)}{x}: \text{Posons}
\begin{cases}
X=x+1\\
x=0
\end{cases}\implies
\begin{cases}
X=x+1\\
x=X-1
\end{cases}\implies
\begin{cases}
\text{Si x}\rightarrow 0\\ \text{alors }\\ X\rightarrow 1
\end{cases}
\]

\[ \text{Donc:}\lim_{X \to 1} \frac{\ln(X)}{X-1}=1 \]

%\[
%b) \lim_{x \to 0} \frac{e^x - 1}{x}\quad \textbf{1,5 pt} 
%\]


%2) Donner les primitives des fonctions suivantes :

%\[
% f(x) = ( \exp \circ f)' \quad\quad\quad\quad f(x) = \frac{f'(x)}{f(x)}  \quad \textbf{1,5 pt}
%\]
\end{document}
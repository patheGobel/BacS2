\documentclass[12pt]{article}
\usepackage{stmaryrd}
\usepackage{graphicx}
\usepackage[utf8]{inputenc}

\usepackage[french]{babel}
\usepackage[T1]{fontenc}
%\usepackage{hyperref}
\usepackage[colorlinks=true, linkcolor=blue, urlcolor=blue, citecolor=blue]{hyperref}
\usepackage{verbatim}

\usepackage{color, soul}

\usepackage{pgfplots}
\pgfplotsset{compat=1.15}
\usepackage{mathrsfs}

\usepackage{amsmath}
\usepackage{amsfonts}
\usepackage{amssymb}
\usepackage{tkz-tab}

\usepackage{tikz}
\usetikzlibrary{arrows, shapes.geometric, fit}


\usepackage[margin=2cm]{geometry}
\usepackage{eso-pic}         % Pour ajouter des éléments en arrière-plan

% Commande pour ajouter du texte en arrière-plan
\AddToShipoutPicture{
    \AtTextCenter{%
        \makebox[0pt]{\rotatebox{45}{\textcolor[gray]{0.9}{\fontsize{5cm}{5cm}\selectfont Pathé BA}}}
    }
}

\begin{document}

\begin{minipage}{0.8\textwidth}
	Pathé BA                          
\end{minipage}
\begin{minipage}{0.8\textwidth}
	BAC 2023
\end{minipage}

\begin{center}
\textbf{{\underline{\textcolor{green}{Premier Groupe Correction}}}}
\end{center}
\section*{\textcolor{green}{\underline{Correction Exercice 1} (3 points) :}}
1) Soit \(a\) un nombre rationnel strictement positif et \(n\) un entier naturel. Donner les limites suivantes :

a) 
\[ \lim_{x \to 0} \frac{\ln(x+1)}{x}=1 \]
\[ \textcolor{green}{\underline{\text{Preuve}}} \]

\[
\lim_{x \to 0} \frac{\ln(x+1)}{x}: \text{Posons}
\begin{cases}
X=x+1\\
x=0
\end{cases}\implies
\begin{cases}
X=x+1\\
x=X-1
\end{cases}\implies
\begin{cases}
\text{Si x}\rightarrow 0\\ \text{alors }\\ X\rightarrow 1
\end{cases}
\]

\[
 \text{Donc:}\lim_{X \to 1} \frac{\ln(X)}{X-1}=1 \textbf{ 0,75 pt}
\]

b)

\[
 \lim_{x \to 0} \frac{e^x - 1}{x}=1 
\]
\[ \textcolor{green}{\underline{\text{Preuve}}} \]

\[
\lim_{x \to 0} \frac{e^x - 1}{x}: \text{Posons}
f(x)=e^{x}\implies
\lim_{x \to 0} \frac{f(x) - f(0)}{x-0}=f'(0)=1
\]

\[
 \lim_{x \to 0} \frac{e^x - 1}{x}=1 \textbf{ 0,75 pt}
\]

2) Donnons les primitives des fonctions suivantes :

a)
\[
( \exp \circ f)f' 
\]

\[
\text{Soit f tel que }f(x)=( \exp \circ f)f' \text{ de primitive } F \text{ donc: } F(x)=\exp\circ f
\]

\[
\textcolor{green}{\boxed{ F(x)=\exp\circ f }} \textbf{ 0,75 points}
\]

b)
\[
\frac{f'(x)}{f(x)}
\]

\[
\text{Soit f tel que }f(x)=\frac{f'(x)}{f(x)} \text{ de primitive } F \text{ donc: } F(x)=\ln(|f(x)|
\]

\[
\textcolor{green}{\boxed{ F(x)=\ln(|f(x)|) }} \textbf{ 0,75 points}
\]
\section*{\textcolor{green}{\underline{Correction Exercice 2} (4 points) }}
Un jeune agriculteur décide de pratiquer la culture sous serre dans son champ. Il place dans un repère orthonormal \( (O; \vec{u}, \vec{v}) \) deux points \( A \) et \( B \) dont les affixes respectives \( z_A \) et \( z_B \) sont des racines du polynôme \( P \) défini par :
\[
P(z) = 2z^3 - 3(1+i)z^2 + 4iz + 1 - i \text{ où } z\in\mathbb{Z}
\]
Il souhaite cultiver dans l'ensemble des points \( M \) de son plan de représentation tels que :
\[
||\overrightarrow{MA} + \overrightarrow{MB} + 2\overrightarrow{MO}|| \leq 2
\]\\
qui contient un point du segment \( [AB] \).

1) Vérifions que \(1\) et \(i\) sont des racines de \(P\). \\
\[
\text{1 et $i$ sont des racines de $P(z)$ ssi  }P(1) = 0 \text{ et } P(i) = 0
\]

\begin{align*}
P(1) &= 2 - 3(1+i) + 4i + 1 - i\\
		&= 2 - 3-3i + 4i + 1 - i\\
		&=0
\end{align*}

\begin{align*}
P(i) &=2i^3 - 3(1+i)i^2 + 4i.i + 1 - i\\
		&=-2i + 3+3i - 4 + 1 - i\\
		&=0
\end{align*}
Donc \(1\) et \(i\) sont bien racines de $P$

2) Déterminons le polynôme \( g(z) \) tel que \( P(z) = (z-1)(z-i)g(z) \). \\

\begin{tabular}{|c|c|c|c|c|}
\hline
&$2$ & $-3(1+i)$ & $4i$ & $1-i$\\
\hline
$1$&&$2$&$-1-3i$&-$1+i$\\
\hline
&$2$&$-1-3i$&$-1+i$&$0$\\
\hline
$i$&&$2i$&$1-i$&\\
\hline
&$2$&$-1-i$&$0$&\\
\hline
\end{tabular}

\[
\textcolor{green}{\boxed{ g(z)=2z-1-i }} \textbf{ 0,5 points}
\]
3) Résolvons dans \( \mathbb{C} \) l'équation \( P(z) = 0 \). \\

\begin{align*}
p(z)=0&\implies (z-1)(z-i)(2z-1-i)=0\\
&(z-1)=0 \text{ ou }(z-i)=0 \text{ ou } (2z-1-i)=0\\
&z=1 \text{ ou }z=i \text{ ou } z=\frac{1+i}{2}
\end{align*}
\[
\textcolor{green}{\boxed{ S=\left\lbrace 1,i,\frac{1+i}{2} \right\rbrace  }} \textbf{ 0,5 points}
\]
4) On pose\( z_A = 1 \), \( z_B = i \), \( z_C = \frac{1}{2}+\frac{1}{2}i \).
\begin{itemize}
\item[a)] Plaçons les points \( A \), \( B \) et \( C \) d'affixes respectives \( z_A\), \( z_B \) et \( z_C \) dans le repère orthonormal \( (O; \vec{u}, \vec{v}) \) en choisissant comme unité graphique 4 cm. \\ 
\newpage
\begin{figure}[h]
\centering
\includegraphics[width=0.8\textwidth]{repère.png}
\caption{Courbe de (Cf)}
\label{fig:monimage}
\end{figure}
\href{https://www.geogebra.org/classic/gh6vrrnw}{Clique ici pour voir la figure sur géogébra}

\item[b)] Démontrons que \( C \) est le milieu de \( [AB] \)

\( C \) est le milieu de \( [AB] \) ssi $\begin{pmatrix}x_{C} \\ \\ y_{C}\end{pmatrix}=
\begin{pmatrix}\frac{x_{B}-x_{A}}{2} \\ \\ \frac{y_{B}-y_{A}}{2}\end{pmatrix}$

En effet, $\begin{pmatrix}\frac{x_{B}-x_{A}}{2} \\ \\ \frac{y_{B}-y_{A}}{2}\end{pmatrix}=
\begin{pmatrix}\frac{1}{2} \\ \\ \frac{1}{2}i\end{pmatrix}$

Or \( z_C = \frac{1+i}{2} \) donc \( C \) est bien le milieu de \( [AB] \)

Démontrons que \( C \) appartient à l'ensemble \( (E) \). \\

\( C \) appartient à l'ensemble \( (E) \) ssi 
$
||\overrightarrow{CA} + \overrightarrow{CB} + 2\overrightarrow{CO}|| \leq 2
$

$
\overrightarrow{CA}=\begin{pmatrix}1-\frac{1}{2} \\ \\ 0-\frac{1}{2}i\end{pmatrix},
\overrightarrow{CB}=\begin{pmatrix}0-\frac{1}{2} \\ \\ i-\frac{1}{2}i\end{pmatrix},
\overrightarrow{CO}=\begin{pmatrix}0-\frac{1}{2} \\ \\ 0-\frac{1}{2}i\end{pmatrix}
$

\begin{align*}
||\overrightarrow{CA} + \overrightarrow{CB} + 2\overrightarrow{CO}|| \leq 2 &\implies
||1-\frac{1}{2}-\frac{1}{2}-\frac{1}{2}-\frac{1}{2}i+i-\frac{1}{2}i-\frac{1}{2}i|| \leq 2\implies
||-\frac{1}{2}-\frac{1}{2}i|| \leq 2 \\
&\implies \sqrt{\left( -\frac{1}{2} \right)^{2}+\left( -\frac{1}{2} \right)^{2} } \leq 2\implies
\frac{\sqrt{2}}{2}\leq 2\quad \text{Cqfd.}
\end{align*}
\text{Donc \( C \) est bien un point de l'ensemble \( (E) \)}

\item[c)] Déterminons l'affixe \( z_{G} \) du point \( G \), barycentre du système \( \{(A,1), (B,1), (O,2)\} \),\\ puis placer \( G \).


Soit \(G\begin{pmatrix}x \\y \\ \end{pmatrix} \) barycentre du système \( \{(A,1), (B,1), (O,2)\} \) donc
\( \overrightarrow{AG}=\frac{1}{4}\overrightarrow{AB}+\frac{1}{2}\overrightarrow{AO}\implies  \)

\begin{align*}
\begin{pmatrix}x-1 \\y-0 \\ \end{pmatrix} &=\frac{1}{4}\begin{pmatrix}0-1 \\1-0 \\ \end{pmatrix}+\frac{1}{2}\begin{pmatrix}-1 \\0 \\ \end{pmatrix} \\
&=\begin{pmatrix}-\frac{1}{4} \\ \frac{1}{4} \\ \end{pmatrix}+\begin{pmatrix}-\frac{1}{2} \\0 \\ \end{pmatrix}\\
&=\begin{pmatrix}-\frac{3}{4} \\ \frac{1}{4} \\ \end{pmatrix}
\end{align*}
Donc \( z_{G}=-\frac{3}{4}+\frac{1}{4}i \)
\[
\textcolor{green}{\boxed{ z_{G}=-\frac{3}{4}+\frac{1}{4}i }} \textbf{ 0,5 points}
\]
\end{itemize}

5) Déterminons puis construisons l'ensemble \( E \) des points \( M \) du plan tels que
\[
||\overrightarrow{MA} + \overrightarrow{MB} + 2\overrightarrow{MO}|| \leq 2
\]
Soit G barycentre de (A,1);(B,1);(0,2)
\[
||\overrightarrow{MA} + \overrightarrow{MB} + 2\overrightarrow{MO}|| \leq 2\implies
||4\overrightarrow{MG} +\overrightarrow{GA}+\overrightarrow{GB} + 2\overrightarrow{GO}|| \leq 2
\] 
\[
||\overrightarrow{MA} + \overrightarrow{MB} + 2\overrightarrow{MO}|| \leq 2\implies
||4\overrightarrow{MG}|| \leq 2\implies MG=\frac{1}{2}
\] 
\[\textcolor{green}{ \text{L'ensemble des points $M$ est un disque de cente $G$} }\]
\begin{figure}[h]
\centering
\includegraphics[width=0.8\textwidth]{disqueBacS22023.png}
\caption{Courbe de (Cf)}
\label{fig:monimage}
\end{figure}\\
\href{https://www.geogebra.org/classic/x9c5uduz}{Clique ici pour voir le disque}

8) Le jeune agriculteur atteindra bien ses objectifs
\section*{\textcolor{green}{\underline{Correction Exercice 3} (4 points) }}
\[
\text{On considère la suite numérique } (u_n)_{n \in \mathbb{N}} \text{ définie par :}
\begin{cases}
U_{0}=6\\
U_{n+1}=\frac{1}{U_{n}}+\frac{3}{4}U_{n}, n\in\mathbb{N}
\end{cases}
\]
\begin{enumerate}
\item[1)] Calculons \( u_1 \) et \( u_2 \).\hfill\textbf{(0.5pt)}
\begin{itemize}
\item Calcule de $u_{1}$:
\begin{align*}
u_{1}&=\frac{1}{u_{0}}+\frac{3}{4}u_{0}\\
		 &=\frac{1}{6}+\frac{3}{4}6\\
		 &=\frac{1}{6}+\frac{9}{2}\\
		 &=\frac{28}{6}\\
\end{align*}
\[\text{Donc }\textcolor{green}{\boxed{u_{1}=\frac{14}{3}}}\]
\item Calcule de $u_{2}$: 
\begin{align*}
u_{2}&=\frac{1}{\frac{14}{3}}+\frac{3}{4}\times\frac{14}{3}\\
		 &=\frac{3}{14}+\frac{14}{4}\\
		 &=\frac{208}{56}\\
		 &=\frac{26}{7}\\
\end{align*}
\[\text{Donc }\textcolor{green}{\boxed{u_{2}=\frac{26}{7}}}\]
\end{itemize}
\item[2)] Démontrons par récurrence que: \( \forall n\in\mathbb{N},\quad u_n \geq \sqrt{3} \).\hfill\textbf{(01pt)}

\textbf{Initialisation:} \\
\text{au rang} $0$: $u_{0}=6\geq \sqrt{3}$\\
\text{au rang} $1$: $u_{1}=\frac{26}{7}\geq \sqrt{3}$\\
\textbf{Hérédité:} \\
Supposons que la propiété est vraie au rang $n$ c'est-à-dire \( \forall n \in \mathbb{N}\quad u_{n} \geq \sqrt{3} \) et montrons que \( \forall n \in \mathbb{N}\quad u_{n+1} \geq \sqrt{3} \)\\
\textbf{En langage mathématique plus condensé:}\\
Montrons que \( \forall n \in \mathbb{N}\), si \(u_{n} \geq \sqrt{3}\) alors \(u_{n+1} \geq \sqrt{3}\)

En supposant \(u_{n} \geq \sqrt{3}\), montrer \(u_{n+1} \geq \sqrt{3}\) revient à montrer \(u_{n+1}-\sqrt{3} \geq 0\).\\
\(u_{n+1}-\sqrt{3} \geq 0\implies\frac{1}{u_{n}}+\frac{3u_{n}^{2}}{4u_{n}}-\frac{\sqrt{3}u_{n}}{u_{n}} \geq 0 \implies \frac{1+3u_{n}^{2}-\sqrt{3}u_{n}}{u_{n}}\)

le signe de \( \frac{1+3u_{n}^{2}-\sqrt{3}u_{n}}{u_{n}} \)  dépend du némérateur car
\( \forall n \in \mathbb{N}\quad u_{n} \geq \sqrt{3}\).

Cherchons le signe du némérateur.

En effet, comme \(u_{n} \geq \sqrt{3}\) donc \(u_{n}^{2} \geq \sqrt{3}u_{n}\) donc \(3u_{n}^{2} \geq \sqrt{3}u_{n}\)
donc \(1+3u_{n}^{2} \geq \sqrt{3}u_{n}\)

donc \(1+3u_{n}^{2}-\sqrt{3}u_{n} \geq  0\) cqfdp

comme \(u_{n} \geq \sqrt{3}\) donc \(u_{n} \geq 0\) donc \(\frac{1}{u_{n}} \geq 0\) 
\[
\begin{cases}
\frac{1}{u_{n}}\geq 0\\
1+3u_{n}^{2}-\sqrt{3}u_{n} \geq  0
\end{cases} \text{Par produit membre à membre, } \frac{1+3u_{n}^{2}-\sqrt{3}u_{n}}{u_{n}} \geq  0 \text{ cqfd}
\]
\textcolor{green}{Donc la proprié est vrai au rang \(n+1\) d'où \(u_{n} \geq \sqrt{3}\)}
\item[3)] Soit $f$ la fonction définie sur $]0, +\infty[$ par \( f(x) = \frac{1}{x} + \frac{3}{4}x \).
\begin{enumerate}
\item[a)] Etudions le sens de variations de $f$.\hfill\textbf{(01pt)}

Calculons la dérivée de \( f(x)\)

\( f'(x) = -\frac{1}{x^{2}} + \frac{3}{4} \implies f'(x) = \frac{3x^{2}-4}{4x^{2}}\)

\text{le signe de \(f'\) dépend du numérateur}

 or 
 
\textcolor{green}{\(\forall x\in]0, \frac{2\sqrt{3}}{3}] \), \(f'(x)\leq 0\) donc f est strictement  décroissante.}

\textcolor{green}{\(\forall x\in[\frac{2\sqrt{3}}{3}, +\infty[ \), \(f'(x)\geq 0\) donc f est strictement croissante.}
\item[b)] Déduidons-en par récurrence que $(u_n)_{n \in \mathbb{N}}$ est strictement décroissante.\hfill\textbf{(0,5pt)}

\textbf{Initialisation:} \\
\text{au rang} $0$: $u_{0}=6$\\
\text{au rang} $1$: $u_{1}=\frac{26}{7}$ donc $u_{0} > u_{1}$\\
\textbf{Hérédité:} \\
Supposons que la propiété est vraie au rang $n$ c'est-à-dire \( \forall n \in \mathbb{N}\quad u_{n} > u_{n+1} \) et montrons que \( \forall n \in \mathbb{N}\quad u_{n+1} > u_{n+2} \)\\
\textbf{En langage mathématique plus condensé:}\\
Montrons que \( \forall n \in \mathbb{N}\), si \(u_{n} > u_{n+1}\) alors \(u_{n+1} > u_{n+2}\)

(L'ojectif ici sera d'utiliser le fait que f est croissant sur \( [\frac{2\sqrt{3}}{3}, +\infty[ \), en montrant que  \(u_{n}\in [\frac{2\sqrt{3}}{3}, +\infty[ \) et \(u_{n+1}\in [\frac{2\sqrt{3}}{3}, +\infty[ \)) car si deux nombres appartiennent dans un même intervalle, et qu'on les appliquent à une même fonction croissante, elles produisent deux nombres conservant le même sens d'inégalité.

En effet, d'après la reccurence précédente, on a : \(u_{n} \geq \sqrt{3}\) et \(u_{n+1} \geq \sqrt{3}\)

Comme \( \sqrt{3} \geq \frac{2\sqrt{3}}{3} \)

donc 

\(u_{n} \geq \frac{2\sqrt{3}}{3}\) et \(u_{n+1} \geq \frac{2\sqrt{3}}{3} \implies\) 
\(u_{n}\in [\frac{2\sqrt{3}}{3}, +\infty[ \) et \(u_{n+1}\in [\frac{2\sqrt{3}}{3}, +\infty[ \)

Comme \(f\) est croissante sur \([\frac{2\sqrt{3}}{3}, +\infty[ \)

donc 

\( f(u_{n})>f(u_{n+1}) \) 

or 

\( f(u_{n})=u_{n+1} \) et \( f(u_{n+1})= u_{n+2}\)

donc \( u_{n+1} > u_{n+2} \)

\textcolor{green}{Donc la proprié est vrai au rang \(n+1\) d'où \( u_{n+1} > u_{n+2} \)}

\textcolor{green}{Donc \( u_{n} \) est une suite décroissante.}
\end{enumerate}
%\item[4)] Montrons que $(u_n)_{n \in \mathbb{N}}$ est convergente et déterminons sa limite.\hfill\textbf{(01pt)}

%\textbf{Convergence:}

%\textbf{Déterminons sa limite:}
\end{enumerate}

\end{document}
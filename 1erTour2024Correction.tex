\documentclass[12pt]{article}
\usepackage{stmaryrd}
\usepackage{graphicx}
\usepackage[utf8]{inputenc}

\usepackage[french]{babel}
\usepackage[T1]{fontenc}
\usepackage{hyperref}
\usepackage{verbatim}

\usepackage{color, soul}

\usepackage{pgfplots}
\pgfplotsset{compat=1.15}
\usepackage{mathrsfs}

\usepackage{amsmath}
\usepackage{amsfonts}
\usepackage{amssymb}
\usepackage{tkz-tab}

\usepackage{tikz}
\usetikzlibrary{arrows, shapes.geometric, fit}


\usepackage[margin=2cm]{geometry}

\begin{document}

\begin{minipage}{0.8\textwidth}
	Pathé BA                          
\end{minipage}
\begin{minipage}{0.8\textwidth}
	BAC 2024
\end{minipage}

\begin{center}
\textbf{{\underline{\textcolor{green}{Premier Groupe Correction}}}}
\end{center}
\section*{\textcolor{green}{\underline{Exercice 1} (5 points) :}}
\begin{enumerate}
\item
\textbf{Module et argument}
\begin{enumerate}
\item Calculons \(z_C - z_B\) et \(z_A - z_B\).

\begin{align*}
    z_C - z_B &= (1 + 2i) - (-2) \\
              &= 1 + 2i + 2 \\
              &= 3 + 2i
\end{align*}

\begin{align*}
    z_A - z_B &= (-3i) - (-2) \\
              &= -3i + 2 \\
              &= 2 - 3i
\end{align*}

Calculons le quotient :

\[
\frac{z_C - z_B}{z_A - z_B} = \frac{3 + 2i}{2 - 3i}=i
\]

	\textbf{Module} :
\[
\frac{z_C - z_B}{z_A - z_B} = \frac{3 + 2i}{2 - 3i}=i\textbf{ 0,25 pt}
\]

\[
\textcolor{green}{\boxed{\mid\frac{z_C - z_B}{z_A - z_B}\mid = |i|=1}}\textbf{ 0,25 pt}
\]

	\textbf{Argument} :

\[
\textcolor{green}{\boxed{\arg(\frac{z_C - z_B}{z_A - z_B})=\arg(i) = \frac{\pi}{2}}}\textbf{ 0,25 pt}
\]

\textbf{Conclusion}

\textcolor{green}{Le module du quotient \(\frac{z_C - z_B}{z_A - z_B}\) est \(1\) et un argument est \(\frac{\pi}{2}\)}.
\item \textbf{Nature du triangle ABC.}

$\mid\frac{z_C - z_B}{z_A - z_B}\mid=1 \implies \frac{BC}{BA}=1\implies BC=BA$.
\begin{align*}
\arg\left( \frac{z_C - z_B}{z_A - z_B}\right) &=\arg\left( z_C - z_B \right) - \arg\left( z_A - z_B \right)\\
&=\left( \vec{u}, \overrightarrow{BC} \right)-\left( \vec{u}, \overrightarrow{BA} \right)\\
&=\left( \overrightarrow{BA}, \vec{u} \right)+\left( \vec{u}, \overrightarrow{BC} \right)\\
&=\left( \overrightarrow{BA}, \overrightarrow{BC} \right)\\
&=\frac{\pi}{2}
\end{align*}
\[
\textcolor{green}{\boxed{\left( \overrightarrow{BA}, \overrightarrow{BC} \right)=\frac{\pi}{2}}}\textbf{ 0,25 pt}
\]
\textbf{Conclusion}

\textcolor{green}{Comme  BA=BC et $\left( \overrightarrow{BA}, \overrightarrow{BC} \right)=\frac{\pi}{2}$}

\textcolor{green}{ABC est un triangle rectangle et isocèle en B}.\textbf{ 0,25 pt $\times$ 3}

\item Déterminons l'affixe de $Z_{D}$

Comme  BADC est un carré donc $\overrightarrow{BA}=\overrightarrow{CD}$.

$\overrightarrow{BA}=\overrightarrow{CD}\implies Z_{A}-Z_{B}=Z_{D}-Z_{C}\implies Z_{D}=Z_{A}-Z_{B}+Z_{C}\implies Z_{D}=3-i$

\[
\textcolor{green}{\boxed{Z_{D}=3-i}}\textbf{ 0,5 pt}
\]

\item Montrons que les points A, B, C et D appartiennent à un même cercle dont on précisera le centre et le rayon.

Comme BADC est un carré donc les points A, B, C et D sont sur un même cerlce donc [BD] est un diamètre ce cercle. 

Un rayon de ce cercle : $r=IB=\frac{BD}{2}$ où I est le centre de ce cerlce

\begin{align*}
\text{\textcolor{green}{Le diamètre}} : BD&=|z_{D}-z_{B}|\\
	&=|3-i-(-2)|\\
	&=|5-i|\\
	&=\sqrt{26}\\
\textcolor{green}{\boxed{BD=\sqrt{26}}}
\end{align*}

\begin{align*}
\text{\textcolor{green}{Le rayon }} : r&=IB=\frac{BD}{2}\\
&=\frac{\sqrt{26}}{2}\textbf{ 0,25 pt}\\
\textcolor{green}{\boxed{r=\frac{\sqrt{26}}{2}}}
\end{align*}

\begin{align*}
\text{\textcolor{green}{Le centre }} : Z_{I}&=\frac{Z_{D}+Z_{B}}{2}\\
&=\frac{3-i+(-2)}{2}\\
&=\frac{1-i}{2}\\
&=\frac{1}{2}-\frac{1}{2}i \\
\textcolor{green}{\boxed{Z_{I}=\frac{1}{2}-\frac{1}{2}i}}\textbf{ 0,25 pt}
\end{align*}

Finalement, A, B, C et D sont sur un même cercle $\mathcal{C}\left(I\begin{pmatrix} \frac{1}{2} \\ -\frac{1}{2}\end{pmatrix}\;;\ \frac{\sqrt{26}}{2}\right)$

\begin{center}
\textbf{\underline{D'autres Approches}}
\end{center}
\begin{enumerate}
\item 
$\text{A, B, C et D appartiennent à un même cercle si :}\left({\overrightarrow {CA}},{\overrightarrow {CB}}\right)=\left({\overrightarrow {DA}},{\overrightarrow {DB}}\right)\mod \pi $

\textbf{ou}
\item
$\text{A, B, C et D appartiennent à un même cercle si :}\arg \left({\frac {c-b}{c-a}}\right)=\arg \left({\frac {d-b}{d-a}}\right)\mod \pi $

\textbf{ou}
\item
$\text{A, B, C et D appartiennent à un même cercle si :}\left[ a,b,c,d \right] =\left({\frac {c-b}{c-a}}\right):\left({\frac {d-b}{d-a}}\right) \text{est réel}
$
\end{enumerate}
\begin{center}
\textbf{\underline{Application :}}
\end{center}
\begin{enumerate}

\item 
$\left({\overrightarrow {CA}},{\overrightarrow {CB}}\right)=\left({\overrightarrow {DA}},{\overrightarrow {DB}}\right)\mod \pi $

\textbf{ou}
\item 
$\arg \left({\frac {c-b}{c-a}}\right)=\arg \left({\frac {d-b}{d-a}}\right)\mod \pi $

\[
\arg \left({\frac {1+2i-(-2)}{1+2i-(-3i)}}\right)=\arg \left({\frac {3-i-(-2)}{3-i-(-3i)}}\right)\mod \pi
\]

\[
\arg \left({\frac {3+2i}{1+5i}}\right)=\arg \left({\frac {5-i}{3+2i}}\right)\mod \pi
\]

\[
\arg \left(3+2i\right)-\arg\left(1+5i\right)=\arg \left(5-i\right)-\arg\left(3+2i\right)\mod \pi
\]
\textbf{ou}
\item 
\[
\text{A, B, C et D appartiennent à un même cercle si :}\left[ a,b,c,d \right] =\left({\frac {c-b}{c-a}}\right):\left({\frac {d-b}{d-a}}\right) \text{est réel}
\]

\begin{align*}
\left({\frac {c-b}{c-a}}\right):\left({\frac {d-b}{d-a}}\right)&=\frac {3+2i}{1+5i} \div \frac {5-i}{3+2i}\\
																&=\frac {3+2i}{1+5i}\times \frac {3+2i}{5-i}\\																		&=\frac {(3+2i)^{2}}{(1+5i)(5-i)}\\
																&=\frac {9-4+24i}{5-i+25i+5}\\
																&=\frac {5+12i}{10+24i}\\
																&=\frac {5+12i}{2(5+12i)}\\
																&=\frac {1}{2}
\end{align*}
\[
\textcolor{green}{\boxed{\left({\frac {c-b}{c-a}}\right):\left({\frac {d-b}{d-a}}\right)=\frac {1}{2}}}\textbf{ 0,25 pt}
\]
\[
\textcolor{green}{\textbf{Donc A, B, C et D sont sur le même cerlce}}
\]

\end{enumerate}

\end{enumerate}
\item 
\begin{enumerate}
\item Montrer $z'=(1+i)z+2-i$ \textbf{ 0,5 pt}
\[\begin{cases}
x'=x-y+2\\
y'=x+y-1
\end{cases}\]
\item Elements caractéristiques
$\theta=\frac{\pi}{4}; k=\sqrt{2}; \omega=1+2i$ \textbf{ 0,25 pt $\times$ 2}
\item Image de (D): $x+y+1=0$

Soit  
\[
\begin{cases}
E\begin{pmatrix} -1 \\ 
0\end{pmatrix}\in (D)\\ \\
F\begin{pmatrix} 0 \\ -1\end{pmatrix}\in (D)
\end{cases} \implies 
\begin{cases}
E'\begin{pmatrix} -3 \\ -2\end{pmatrix}\\ \\
F'\begin{pmatrix} 1 \\ -2\end{pmatrix}
\end{cases}
\]

\[
\det(A_y) = \begin{vmatrix}
4 & x-1 \\
0 & y+2
\end{vmatrix} = (4)(y+2) = 4y+8=0\implies y=-2
\]

(D') est la droite (E'F'): y=-2 $\textbf{0,5}$


\item Ensemble des point M tel que:

$\mid (1+i)z+2-i\mid=2 \implies \mid (1+i)\left( z+\frac{2-i}{1+i}\right) \mid=2 \implies \sqrt{2}\mid z+\left( \frac{1}{2}-\frac{3}{2}i\right) \mid=2 \implies $

$\mid z-\left( -\frac{1}{2}+\frac{3}{2}i\right) \mid=\sqrt{2}$

Soit M un point d'affixe z et P un point d'affixe $-\frac{1}{2}+\frac{3}{2}i$

Donc $ \mid z-\left( -\frac{1}{2}+\frac{3}{2}i\right) \mid=\sqrt{2} \implies \mid z_{M}- z_{A} \mid=\sqrt{2} \implies AM=\sqrt{2}$


D'où $\mid z-(\frac{-1}{2}+\frac{3}{2}i)\mid$ Ainsi, on a $ C\left( \begin{pmatrix} \frac{-1}{2} \\ \frac{3}{2}\end{pmatrix}, \sqrt{2}\right) $
\end{enumerate}
\end{enumerate}

\textbf{Maintenant, je revendrai avec une deuxième méthode utilisant l'équation du cercle}

\section*{\textcolor{green}{\underline{Exercice 2} (5 points) :}}

Une entreprise fabrique des articles dans deux unités de production notées \( U_1 \) et \( U_2 \). L’unité \( U_1 \) assure 60\% de la production.

On a constaté que :

- 3\% des articles provenant de l’unité \( U_1 \) présentent un défaut de fabrication.

- 8\% des articles provenant de l’unité \( U_2 \) présentent un défaut de fabrication.

L’entreprise envisage de mettre en place un test de contrôle de ces articles avant leur mise en vente. Ce contrôle détecte et élimine 82\% des articles défectueux, mais il élimine également à tort 4\% des articles non défectueux. Les articles non éliminés sont alors mis en vente.

L’entreprise souhaite que moins de 1\% des articles vendus soient défectueux.

À l’aide des informations ci-dessus et des outils mathématiques au programme :
\begin{enumerate}
    \item Démontrer que 5\% des articles produits présentent un défaut de fabrication. \textbf{(02 points)}
    \item En prenant au hasard un article fabriqué, montrer que la chance que cet article soit mis en vente après contrôle est de 0,921. \textbf{(02 points)}
    \item Vérifier si ce contrôle permet à l’entreprise de réaliser son souhait. \textbf{(01 point)}
\end{enumerate}

\section*{\textcolor{green}{\underline{Correction Exercice 2} (5 points) :}}
$P_{U_{1}}(D)=\frac{60}{100}=\frac{3}{5}$

$P_{U_{1}}(D)=\frac{60}{100}=\frac{3}{5}$

\begin{center}
\definecolor{ududff}{rgb}{0.30196078431372547,0.30196078431372547,1}
\definecolor{cqcqcq}{rgb}{0.7529411764705882,0.7529411764705882,0.7529411764705882}
\begin{tikzpicture}[line cap=round,line join=round,>=triangle 45,x=1cm,y=1cm]
%\draw [color=cqcqcq,, xstep=1cm,ystep=1cm] (-8.3,-6.45) grid (15.34,5.57);
%\clip(-8.3,-6.45) rectangle (15.34,5.57);
\draw [line width=2pt] (-4,0)-- (0,2);
\draw [line width=2pt] (0,2)-- (3,4);
\draw [line width=2pt] (0,2)-- (3,0);
\draw [line width=2pt] (-4,0)-- (0,-2);
\draw [line width=2pt] (3,-1)-- (0,-2);
\draw [line width=2pt] (0,-2)-- (3,-4);
\begin{scriptsize}
\draw [fill=ududff] (0,2) circle (2.5pt);
\draw[color=ududff] (0.16,2.42) node {$U_{1}$};
\draw[color=black] (-1.78,2) node {$60\%$};
\draw [fill=ududff] (3,4) circle (2.5pt);
\draw[color=ududff] (3.16,4.42) node {$D$};
\draw[color=black] (1.74,3.5) node {$3\%$};
\draw [fill=ududff] (3,0) circle (2.5pt);
\draw[color=ududff] (3.16,0.42) node {$\overline{D}$};
\draw[color=black] (1.4,0.8) node {$97\%$};
\draw [fill=ududff] (0,-2) circle (2.5pt);
\draw[color=ududff] (0.16,-1.58) node {$U_{2}$};
\draw[color=black] (-2.08,-1.5) node {$40\%$};
\draw [fill=ududff] (3,-1) circle (2.5pt);
\draw[color=ududff] (3.16,-0.58) node {$D$};
\draw[color=black] (1.46,-0.96) node {$8\%$};
\draw [fill=ududff] (3,-4) circle (2.5pt);
\draw[color=ududff] (3.16,-3.58) node {$\overline{D}$};
\draw[color=black] (1.4,-3.5) node {$92\%$};
\end{scriptsize}
\end{tikzpicture}
\end{center}
\begin{align*}
	P(D)&=P(U_{1}\cap D)+P(U_{2}\cap D)\\
		&=P(U_{1})\times P_{U_{1}}(D)+P(U_{2})\times P_{U_{2}}(D)\\
		&=0,03\times 0,6+0,08\times 0,4\\
		&=0,05
\end{align*}
\begin{center}
\boxed{P(D)=0,05}
\end{center}
2)
% Données
 
% Le contrôle détecte et élimine 82% des articles défectueux.
% Le contrôle élimine également à tort 4% des articles non défectueux.
% 5% des articles produits sont défectueux.

\textbf{Données :}
\begin{itemize}
    \item La probabilité qu'un article soit défectueux (\(D\)) est \(0,05\).
    \item La probabilité qu'un article soit non défectueux (\(\overline{D}\)) est \(0,95\).
\end{itemize}

\textbf{Probabilités conditionnelles :}
\begin{itemize}
    \item La probabilité qu'un article défectueux soit éliminé (\(E\)) : \(P(E|D) = 0,82\).
    \item La probabilité qu'un article non défectueux soit éliminé : \(P(E|\overline{D}) = 0,04\).
\end{itemize}
\begin{center}
\definecolor{ududff}{rgb}{0.30196078431372547,0.30196078431372547,1}
\definecolor{cqcqcq}{rgb}{0.7529411764705882,0.7529411764705882,0.7529411764705882}
\begin{tikzpicture}[line cap=round,line join=round,>=triangle 45,x=1cm,y=1cm]
%\draw [color=cqcqcq,, xstep=1cm,ystep=1cm] (-8.3,-6.45) grid (15.34,5.57);
%\clip(-8.3,-6.45) rectangle (15.34,5.57);
\draw [line width=2pt] (-4,0)-- (0,2);
\draw [line width=2pt] (0,2)-- (3,4);
\draw [line width=2pt] (0,2)-- (3,0);
\draw [line width=2pt] (-4,0)-- (0,-2);
\draw [line width=2pt] (3,-1)-- (0,-2);
\draw [line width=2pt] (0,-2)-- (3,-4);
\begin{scriptsize}
\draw [fill=ududff] (0,2) circle (2.5pt);
\draw[color=ududff] (0.16,2.42) node {$D$};
\draw[color=black] (-1.78,2) node {$0,05\%$};
\draw [fill=ududff] (3,4) circle (2.5pt);
\draw[color=ududff] (3.16,4.42) node {$E$};
\draw[color=black] (1.74,3.5) node {$0,82\%$};
\draw [fill=ududff] (3,0) circle (2.5pt);
\draw[color=ududff] (3.16,0.42) node {$\overline{E}$};
\draw[color=black] (1.4,0.8) node {$...$};
\draw [fill=ududff] (0,-2) circle (2.5pt);
\draw[color=ududff] (0.16,-1.58) node {$D$};
\draw[color=black] (-2.08,-1.5) node {$0,95\%$};
\draw [fill=ududff] (3,-1) circle (2.5pt);
\draw[color=ududff] (3.16,-0.58) node {$E$};
\draw[color=black] (1.46,-0.96) node {$0,04\%$};
\draw [fill=ududff] (3,-4) circle (2.5pt);
\draw[color=ududff] (3.16,-3.58) node {$\overline{E}$};
\draw[color=black] (1.4,-3.5) node {$...$};
\end{scriptsize}
\end{tikzpicture}
\end{center}
\textbf{Probabilités complémentaires :}
\begin{itemize}
    \item La probabilité qu'un article défectueux ne soit pas éliminé (\(\overline{E}\)) : 
    \[
    P(\overline{E}|D) = 1 - P(E|D) = 1 - 0,82 = 0,18
    \]
    \item La probabilité qu'un article non défectueux ne soit pas éliminé : 
    \[
    P(\overline{E}|\overline{D}) = 1 - P(E|\overline{D}) = 1 - 0,04 = 0,96
    \]
\end{itemize}

\textbf{Probabilité totale qu'un article soit mis en vente (\(\overline{E}\)) :}
\[
P(\overline{E}) = P(\overline{E}|D) \times P(D) + P(\overline{E}|\overline{D}) \times P(\overline{D})
\]

%En substituant les valeurs :
%\[
%P(\overline{E}) = 0.18 \times 0.05 + 0.96 \times 0.95
%\]

%\[
%P(\overline{E}) = 0.009 + 0.912
%\]

\begin{center}
\textcolor{green}{\boxed{P(\overline{E}) = 0.921}}
\end{center}

\textbf{Conclusion :} La probabilité qu'un article pris au hasard soit mis en vente après contrôle est de \(0,921\).

3. Vérifier si ce contrôle permet à l’entreprise de réaliser son souhait. \textbf{(01 point)}
    
    Nous avons déjà déterminé que la probabilité qu'un article soit mis en vente après contrôle est de 0,921. Maintenant, nous devons calculer la probabilité qu'un article vendu soit défectueux, compte tenu du contrôle.

    \textbf{Probabilité qu'un article vendu soit défectueux:}

    \[
    P_{\overline{E}}(\text{D}) = \frac{P(\text{D}) \times P_{D}(\overline{E})}{P(\overline{E})}
    \]

    Nous savons que:
    \[
    P(\text{D}) = 0.05
    \]
    \[
    P_{D}(\overline{E}) = 0.18
    \]
    \[
    P(\overline{E}) = 0.921
    \]

    En substituant ces valeurs:

    \[
    P_{\overline{E}}(\text{D}) = \frac{0.05 \times 0.18}{0.921}
    \]

    Calculons le numérateur:

    \[
    0.05 \times 0.18 = 0.009
    \]

    Ensuite, divisons par \(P(\text{ V})\):

    \[
    P_{\overline{E}}(\text{D}) = \frac{0.009}{0.921} \approx 0.00977
    \]

    Donc, la probabilité qu'un article vendu soit défectueux est environ \(0.00977\), ce qui représente environ \(0.977\%\).

    Étant donné que \(0.977\%\) est inférieur à \(1\%\), cela signifie que ce contrôle permet à l'entreprise de réaliser son souhait de vendre moins de 1\% d'articles défectueux.
\section*{\textcolor{green}{\underline{CORRECTION DU PROBLEME :} ( 10 points ).}}
\subsection*{ \underline{PARTIE A } ( 2 points ) }
\begin{enumerate}
\item Pour tout $x < 0$, on pose : $u(x)=x+1-e^{-x}$.

Etudions le signe de $1-e^{-x}$ pour $x < 0$.

Supposons que $1-e^{-x} < 0$

$ 1-e^{-x} < 0 \implies 1 < e^{-x} \implies 0 < -x \implies 0 > x \implies x\in \left]  -\infty; 0\right[  $

donc si $x\in \left]  -\infty; 0\right[$ alors $1-e^{-x} < 0$

Déduisons-en que pour tout $x < 0, u(x) < 0 $.

D'après la question précédente, $1-e^{-x} < 0$ comme $x < 0$ donc par somme
\[
\underline{\begin{cases}
1-e^{-x} < 0\\
x < 0
\end{cases}}
\]
\[x+1-e^{-x} < 0\]
D'où $ u(x) < 0 $
\item Pour tout $x > 0,$ on pose : $v(x)=x-1-\ln x$.
\begin{enumerate}
\item[a.] Dressons le tableau de variations de $v$.


\textcolor{green}{\underline{*continuité}}

$v(x)$ est la somme de deux fonctions continue donc continue $\mathbb{R}^{*}_{+}$ donc continue sur $\mathbb{R}^{*}_{+}$

\textcolor{green}{\underline{*dérivabilité}}

$x\mapsto x-1$ est dérivable sur $\mathbb{R}$ en particulier sur $\mathbb{R}^{*}_{+}$

$x\mapsto \ln x$ est dérivable sur $\mathbb{R}^{*}_{+}$

Donc $x\mapsto x-1-\ln x$ est dérivable sur $\mathbb{R}^{*}_{+}$

D'où $v(x)$ est dérivable sur $\mathbb{R}^{*}_{+}$

\textcolor{green}{\underline{*dérivée}}

$v'(x)=1-\frac{1}{x} \implies v'(x)=\frac{x-1}{x}$

\textcolor{green}{\underline{le signe de $v'(x)$}}

Le signe de  $v'$ dépend du numérateur car pour le dénominateur, $\forall x\in \mathbb{R}^{*}_{+}$, x>0
 
--$ \forall x\in \left]0;1\right[, x-1<0$ donc $ v'(x)<0 $ donc $v(x)$ est décroissante sur $\left]0;1\right[$

--$ \forall x\in \left]1;+\infty\right[, x-1<0$ donc $ v'(x)>0 $ donc $v(x)$ est croissante sur $\left]0;+\infty\right[$

\textcolor{green}{\underline{les limites aux bornes de $Dv$}}

En $0$:

\[
\lim_{x\to 0^{+}}v(x)=\lim_{x\to 0^{+}}x-1-\ln x:
\begin{cases}
\lim_{x\to 0^{+}}x-1 = -1 \\
\lim_{x\to 0^{+}}-\ln x =+\infty
\end{cases}
\text{Par somme:}\lim_{x\to 0^{+}}v(x)=+\infty
\]
\[\textcolor{green}{\boxed{\lim_{x\to 0^{+}}v(x)=+\infty}}\]
En $+\infty$:

\[
\lim_{x\to +\infty}v(x)=\lim_{x\to +\infty}x-1-\ln x:
\begin{cases}
\lim_{x\to +\infty}x-1 = +\infty \\
\lim_{x\to +\infty}-\ln x =-\infty
\end{cases}
\text{Par somme forme indéterminée}
\]
Levons l'indétermination:

\[
\lim_{x\to +\infty}x-1-\ln x = \lim_{x\to +\infty}x(1-\frac{1}{x}-\frac{\ln x}{x}):
\begin{cases}
\lim_{x\to +\infty}x = +\infty \\
\lim_{x\to +\infty} 1-\frac{1}{x}-\frac{\ln x}{x} =1
\end{cases}
\text{Par produit, }
\]
\[\textcolor{green}{\boxed{\lim_{x\to +\infty}v(x)=+\infty}}\]

\[\text{Le calcul de v(1):}\textcolor{green}{\boxed{v(1)=0}}\]
\begin{tikzpicture}
 \tkzTabInit{$x$/1,$v'$/1,$v$/2}{$0$,$1$,$+\infty$}
 \tkzTabLine{,-,z,+,}
 \tkzTabVar{+/$+\infty$,-/$0$,+/$+\infty$}
 %\tkzTabVal{2}{3}{0.5}{$\pi$}{0}
 % Add a movable label
 %\node at (3.5, -1.5) {$7$};
\end{tikzpicture}
\item[b.] Déduisons-en le signe de $v(x)$ pour $x > 0 $.

D'après le tableau de variation, $\forall x\in \mathbb{R}_{+}^{*}$, $v(x) \geq 0$
\end{enumerate}
\end{enumerate}
\subsection*{ \underline{PARTIE B } ( 8 points ) }
\[
\text{Soient f la fonction définie par} 
f(x)=
\begin{cases}
xe^{x}-x-1 \quad\quad  \text{ si }  x \leq 0\\
x^{2}-1-2x\ln x \quad \text{ si } x > 0 
\end{cases}
\]
et $C_{f}$ sa courbe représentative dans un repère orthonormé (O;$\vec{i}$;$\vec{j}$) d'unité 1 cm.
\begin{enumerate}
\item
\begin{enumerate}
\item[a.] Montrons que l'ensemble de définition de $f$ est $\mathbb{R}$.\textbf{ 0,5 points}

\[
\text{Posons } 
f(x)=
\begin{cases}
f_{1}(x)=xe^{x}-x-1 \quad\quad  \text{ si }  x \leq 0\\
f_{2}(x)=x^{2}-1-2x\ln x \quad \text{ si } x > 0 
\end{cases}
\]
\item[-] $f_{1}$ existe toujours car $\forall x\in\mathbb{R}$ et $x<0$. \\

Donc $Df_{1}=\mathbb{R}$

\item[-] $f_{2}$ existe ssi et seulement si x > 0 et x > 0.

Donc $Df_{2}=\mathbb{R}_{+}^{*}$
\\\\
D'où $Df=Df_{2}\cup Df_{1}=\mathbb{R}$

finalement, $Df=\mathbb{R}$
\item[b.] Etudions les limites de en $-\infty$ et en $+\infty$.\textbf{ 0,5 points}

\textcolor{green}{\underline{En $-\infty$}}
\[
\lim_{x \to -\infty}f(x)=\lim_{x \to -\infty}xe^{x}-x-1 : 
\begin{cases}
\lim_{x \to -\infty}xe^{x}=\lim_{x \to -\infty}\frac{x}{e^{-x}}=0\\
\lim_{x \to -\infty}-x-1=+\infty
\end{cases}
\text{Par somme}
\]

\[\textcolor{green}{\boxed{\textcolor{green}{\boxed{\lim_{x\to -\infty}v(x)=+\infty}}}}\]

\textcolor{green}{\underline{En $+\infty$}}
\[
\lim_{x \to +\infty}f(x)=\lim_{x \to +\infty}x^{2}-1-2x\ln x : 
\begin{cases}
\lim_{x \to +\infty}x^{2}-1=\lim_{x \to +\infty}x^{2}-1=+\infty\\
\lim_{x \to +\infty}-2x\ln x=-\infty
\end{cases}
\text{Par somme, FI}
\]
\text{Levons l'indétermination}
\begin{align*}
\lim_{x \to +\infty}x^{2}-1-2x\ln x=\lim_{x \to +\infty}x^{2}\left(1-\frac{1}{x^{2}}-\frac{2\ln x}{x} \right):
\begin{cases}
\lim_{x \to +\infty}x^{2}=+\infty\\
\lim_{x \to +\infty}\left(1-\frac{1}{x^{2}}-\frac{2\ln x}{x}\right)=1
\end{cases}
\text{Par pro..., } 
\end{align*}
\[\textcolor{green}{\boxed{\textcolor{green}{\boxed{\lim_{x\to +\infty}v(x)=+\infty}}}}\]
\item[c.]Montrons que la droite $(D)$ d'équation $y=-x-1$ est

 asymptote à $C_{f}$ en $-\infty$. \textbf{ 0,25 points}

\[y=-x-1 \text{ est asymptote à } C_{f} \text{ en } -\infty \text{ ssi } \lim_{x \to -\infty}\left[f(x)-y \right]=0 \]
\[\lim_{x \to -\infty}\left[f(x)-y \right]=\lim_{x \to -\infty}\left[xe^{x}-x-1-(-x-1) \right]=\lim_{x \to -\infty}xe^{x}=0\]
Donc $y$ est asymptote oblique à $ C_{f}$ en $-\infty$

Précisons la position de $(C_{f})$ par rapport à $(D)$ sur $]-\infty, 0[.$ \textbf{ 0,25 points}

Pour ce faire, étudions le signe de $f(x)-y$

En effet, $f(x)-y=xe^{x}-x-1-(-x-1)=xe^{x}$ comme $x < 0 $ donc $xe^{x} < 0$ ainsi, $f(x)-y<0$

D'où  $(C_{f})$ est en dessous de $(D)$ sur $]-\infty, 0[$
\item[d.]Etudions la nature de la branche infinie de $(C_{f})$ en $+\infty$.\textbf{ 0,5 points}

\[\text{Pour ce faire, calculons } \lim_{x \to +\infty}\frac{f(x)}{x} \]

\begin{align*}
\lim_{x \to +\infty}\frac{f(x)}{x}&=\lim_{x \to +\infty}\frac{x^{2}-1-2x\ln x}{x}=\lim_{x \to +\infty}\frac{x^{2}(1-\frac{1}{x^{2}}-2\frac{\ln x}{x})}{x}\\
&=\lim_{x \to +\infty}x^{2}\left( 1-\frac{1}{x^{2}}-2\frac{\ln x}{x}\right) :
\begin{cases}
\lim_{x \to +\infty} x^{2}=+\infty\\
\lim_{x \to +\infty}\left( 1-\frac{1}{x^{2}}-2\frac{\ln x}{x}\right)=1
\end{cases}
\text{Par produit, }\\
\end{align*}
\[\textcolor{green}{\boxed{\textcolor{green}{\boxed{\lim_{x\to +\infty}\frac{f(x)}{x}=+\infty}}}}\]
Donc (Cf) admet une branche parabolique de direction $(ox)$
\end{enumerate}
\item
\begin{enumerate}
\item[a.]Etudions la contuinité de $f$ en  $0$. \textbf{ 0,5 point}
\[\text{ f est continue en 0 ssi }\lim_{x \to 0^{-}} f(x)=\lim_{x \to 0^{+}} f(x)=f(0)\]
En $0^{-}$
\[
\lim_{x \to 0^{-}}f(x)=\lim_{x \to 0^{-}}xe^{x}-x-1:
\begin{cases}
\lim_{x \to 0^{-}}xe^{x}=0\\
\lim_{x \to 0^{-}}-x-1=-1
\end{cases}
\text{ Par somme,}\textcolor{green}{\lim_{x \to 0^{-}}f(x)=-1}
\]
En $0^{+}$
\[
\lim_{x \to 0^{+}}f(x)=\lim_{x \to 0^{+}}x^{2}-1-2x\ln x:
\begin{cases}
\lim_{x \to 0^{+}}x^{2}-1=-1\\
\lim_{x \to 0^{+}}-2x\ln x=0
\end{cases}
\text{ Par somme,}\textcolor{green}{\lim_{x \to 0^{+}}f(x)=-1}
\]
\[f(0)=0e^{0}-0-1=-1 \text{ donc,} f(0)=-1\]
\[\textcolor{green}{\text{ comme }\lim_{x \to 0^{-}} f(x)=\lim_{x \to 0^{+}} f(x)=f(0)\text{ donc f est continue en 0 }}\]
\item[b.]Etudions la dérivabilité de $f$ en $0$. \textbf{ 0,5 point}
\[
\text{ f est derivable en 0 ssi }\lim_{x \to 0^{-}} \frac{f(x)-f(0)}{x}=\lim_{x \to 0^{+}} \frac{f(x)-f(0)}{x}
\]
En $0^{-}$
\[
\lim_{x \to 0^{-}} \frac{f(x)-f(0)}{x}=\lim_{x \to 0^{-}} \frac{xe^{x}-x-1-(-1)}{x}=\lim_{x \to 0^{-}} \frac{xe^{x}-x}{x}=\lim_{x \to 0^{-}} e^{x}-1=0
\]
\[\textcolor{green}{\boxed{\textcolor{green}{\boxed{\lim_{x \to 0^{-}} \frac{f(x)-f(0)}{x}=0}}}}\]
En $0^{+}$
\begin{align*}
\lim_{x \to 0^{+}} \frac{f(x)-f(0)}{x}&=\lim_{x \to 0^{+}} \frac{x^{2}-1-2x\ln x-(-1)}{x}\\
&=\lim_{x \to 0^{+}} \frac{x^{2}-2x\ln x}{x}=\lim_{x \to 0^{+}} x-2\ln x:
\begin{cases}
\lim_{x \to 0^{+}} x=0\\
\lim_{x \to 0^{+}} -2\ln x=+\infty
\end{cases}
\text{ par somme,}
\end{align*}
\[\lim_{x \to 0^{+}} \frac{f(x)-f(0)}{x}=+\infty\]
\[\textcolor{green}{\boxed{\textcolor{green}{\boxed{\lim_{x \to 0^{+}} \frac{f(x)-f(0)}{x}=+\infty}}}}\]

\[\text{ finalement, }\lim_{x \to 0^{-}} \frac{f(x)-f(0)}{x}\neq\lim_{x \to 0^{+}} \frac{f(x)-f(0)}{x} \text{ donc f n'est pas dérivable en 0.}\] 
	Interpretons  graphiquement les résultats obtenus. \textbf{ 0,5 point}
\textcolor{green}{
\begin{align*}
\text{ Comme }\lim_{x \to 0^{-}} \frac{f(x)-f(0)}{x}=0&\text{ donc f est dérivable à gauche de 0}\\ 
													&\text{mais admet une demi-tangente horizontale d'équation }y=-1.
\end{align*}
}
\textcolor{green}{
 \begin{align*}
\text{ Comme }\lim_{x \to 0^{+}} \frac{f(x)-f(0)}{x}=+\infty & \text{ donc f n'est dérivable à droite de 0}\\ 
													&\text{mais admet une demi-tangente verticale orientée vers le haut. }
\end{align*}
}
\end{enumerate}
\item
\begin{enumerate}
\item[a.] Montrons que pour tout $x < 0 $, $f'(x)=u(x)e^{x}$. \textbf{ 0,5 point}

Si $x < 0 $,  $f(x)=xe^{x}-x-1 $
\begin{align*}
f'(x)&=e^{x}+xe^{x}-1\\
	 &=(1+x-e^{-x})e^{x} \text{ or, } u(x)=x+1-e^{-x} \text{ donc }\\
	 &=u(x)e^{x}
\end{align*}
	Déduisons-en le signe de $f'(x)$ sur $]-\infty, 0[.$ \textbf{ 0,25 point}
	
\textcolor{green}{Le signe de $f'$ dépend de celui de $u$ car $\forall x$, $e^{-x}>0$. Or, d'après ce qui précède,
$\forall x<0$, $u(x)<0$ donc $f'(x)<0$}
\item[b.] Montrons que pour tout $ x > 0$, $f'(x)=2v(x).$ \textbf{ 0,5 point}
\begin{align*}
f'(x)&=x^{2}-1-2x\ln x\\
	 &=2x-2\left(\ln x + 1 \right)\\
	 &=2\left(x-\ln x - 1\right)  \text{ or, } v(x)=x-1-\ln x \text{ donc }\\
	 &=2v(x)
\end{align*}
	Déduisons-en le signe de $f'(x)$ sur $]0, +\infty[.$ \textbf{ 0,25 point}
	
\textcolor{green}{Le signe de $f'$ dépend de celui de $v$. Or, d'après ce qui précède,
$\forall x > 0$, $v(x)\geq 0$ donc $f'(x)\geq 0$}
\item[c.] Dressons le tableau de variations de $f$. \textbf{ 0,5 point}

\begin{tikzpicture}
 \tkzTabInit{$x$/1,$f'(x)$/1,$f$/2}{$-\infty$,$1$,$+\infty$}
 \tkzTabLine{,-,z,+,}
 \tkzTabVar{+/$+\infty$,-/$0$,+/$+\infty$}
 \node at (5.2, -1.5) {\textcolor{red}{0}};
 \node at (6, -1.5) {\textcolor{red}{$+\infty$}};
\end{tikzpicture}
\end{enumerate}
\item Traçons $(D)$ et $(C_{f})$ dans le plan muni du repère (O,$\vec{i}$,$\vec{j}$). \textbf{ 0,25 point + 0,5 point}
\begin{figure}[h]
\centering
\includegraphics[width=0.8\textwidth]{courbe.png}
\caption{Courbe de (Cf)}
\label{fig:monimage}
\end{figure}
%\includegraphics[scale=0.8]{c1c2c3.png}
\item
\begin{enumerate}
\item[a.] Soit $g$ la restriction de $f$ à l'intervalle $]0, +\infty[.$

	 Montrons que $g$ admet une bijection réciproque $g^{-1}$ dont on précisera l'ensemble de définition et le
	 sens de variations. \textbf{ 0,25 point + 0,25 point + 0,25 point}

\textcolor{green}{Comme $g$ est la restriction de $f$ sur l'intervalle $]0, +\infty[$ donc $g$ hérite des propriétés de $g$ donc $g$ est continue et strictement croissant donc $g$ admet une bijection de $]0, +\infty[$ vers $g(]0, +\infty[)=]0, +\infty[$.\\
Comme $g$ est bijectif de $]0, +\infty[$ vers $g(]0, +\infty[)=]0, +\infty[$ donc il amet une bijection réproque $g^{-1}$ de $g(]0, +\infty[)$ vers  $]0, +\infty[$\\
$Dg=]0, +\infty[$}
\item[b.] Tracons la courbe representative $C_{g^{-1}}$ de $g^{-1}$ dans le plan muni du repère \\
			(O,$\vec{i}$,$\vec{j}$). \textbf{ 0,25 point}
			
\begin{figure}[h]
\centering
\includegraphics[width=0.8\textwidth]{c1c2c3.png}
\caption{Courbe de (Cf)}
\label{fig:monimage}
\end{figure}
\end{enumerate}
\newpage
\item Soit $\lambda$ un réel strictement négatif.
\begin{enumerate}
\item[a.] Exprimons l'aire $A(\lambda)$ en fonction de $\lambda$ la partie du plan délimitée par les droites d'équations $x=\lambda$, $x=0$, $y=-x-1$ et la courbe $(C_{f})$. \textbf{ 0,25 point}
\begin{align*}
\int_{\lambda}^{0}\left( f(x)-y \right)dx &=\int_{\lambda}^{0}\left( xe^{x}-x-1-(-x-1) \right)dx\\
&=\int_{\lambda}^{0}\left( xe^{x}\right)dx\\
&=\left[\left( x-1\right) e^{x}\right]_{\lambda}^{0} \\
&=\left[\left( 0-1\right) e^{0}-\left( \lambda-1\right) e^{\lambda}\right] \\
&=\left[-1-\lambda e^{\lambda}+e^{\lambda}\right] \\
\text{On sait que }-1-\lambda e^{\lambda}+e^{\lambda}<0 \text{ donc }
\end{align*}

\[\textcolor{green}{\boxed{\boxed{A(\lambda)=-\left( -1-\lambda e^{\lambda}+e^{\lambda} \right)\times 1cm^{2}}}}\]

\item[b.] Déduisons-en $\lim_{\lambda \to -\infty}A(\lambda)$. \textbf{ 0,25 point}
\[
\lim_{\lambda \to -\infty}A(\lambda)=
\lim_{\lambda \to -\infty}-e^{\lambda}+\frac{\lambda}{e^{-\lambda}}+1=+1
\]

\[\textcolor{green}{\boxed{\boxed{\lim_{\lambda \to -\infty}A(\lambda)=1}}}\]
\end{enumerate}
\end{enumerate}
\end{document}
\documentclass[12pt]{article}
\usepackage{stmaryrd}
\usepackage{graphicx}
\usepackage[utf8]{inputenc}

\usepackage[french]{babel}
\usepackage[T1]{fontenc}
\usepackage{hyperref}
\usepackage{verbatim}

\usepackage{color, soul}

\usepackage{pgfplots}
\pgfplotsset{compat=1.15}
\usepackage{mathrsfs}

\usepackage{amsmath}
\usepackage{amsfonts}
\usepackage{amssymb}
\usepackage{tkz-tab}

\usepackage{tikz}
\usetikzlibrary{arrows, shapes.geometric, fit}


\usepackage[margin=2cm]{geometry}

\usepackage{enumitem}

\begin{document}

\begin{minipage}{0.8\textwidth}
	Pathé BA                          
\end{minipage}
\begin{minipage}{0.8\textwidth}
	BAC-S2 2023
\end{minipage}

\begin{center}
\textbf{{\underline{\textcolor{red}{Premier Groupe}}}}
\end{center}
%\section*{\textcolor{red}{\underline{EXERCICE 1:} (5 points) :}}
%\documentclass[a4paper,11pt]{article}
%\usepackage{amsmath,amsfonts,amssymb}
%\usepackage{graphicx}

%\title{Épreuve de Mathématiques \\ Séries S2, S2A, S4, S5}
%\author{Baccalauréat Sénégal - Session 2023}
%\date{}

%\begin{document}

%\maketitle

%\textbf{Durée : 4 heures \hfill Coefficient : 5} \\
%Les calculatrices électroniques non imprimantes avec entrée unique par clavier sont autorisées. Les calculatrices permettant d'afficher des formules ou des tracés de courbes sont interdites.

\section*{\textcolor{red}{\underline{EXERCICE 1} (3 points) }}
1) Soit \(a\) un nombre rationnel strictement positif et \(n\) un entier naturel. Donner les limites suivantes :

\[
a) \lim_{x \to 0} \frac{\ln(x+1)}{x}\quad\quad\quad\quad b) \lim_{x \to 0} \frac{e^x - 1}{x}\quad \textbf{1,5 pt} \]


2) Donner les primitives des fonctions suivantes :

\[
a) ( \exp \circ f)f' \quad\quad\quad\quad  b) \frac{f'(x)}{f(x)}  \quad \textbf{1,5 pt}
\]

\section*{\textcolor{red}{\underline{EXERCICE 2} (4 points) }}
Un jeune agriculteur décide de pratiquer la culture sous serre dans son champ. Il place dans un repère orthonormal \( (O; \vec{u}, \vec{v}) \) deux points \( A \) et \( B \) dont les affixes respectives \( z_A \) et \( z_B \) sont des racines du polynôme \( P \) défini par :
\[
P(z) = 2z^3 - 3(1+i)z^2 + 4iz + 1 - i
\]
Il souhaite cultiver dans l'ensemble des points \( M \) de son plan de représentation tels que :
\[
||\overrightarrow{MA} + \overrightarrow{MA} + 2\overrightarrow{MO}|| < 2
\]\\
qui contient un point du segment \( [AB] \).

\renewcommand{\labelenumi}{\theenumi)}
\begin{enumerate}[label=\arabic*)]
    \item Vérifier que \(1\) et \(i\) sont des racines de \(P\).\hfill\textbf{(0,5pt)}
    \item Déterminer le polynôme \( g(z) \) tel que \( P(z) = (z-1)(z-i)g(z) \).\hfill\textbf{(0,5pt)}
    \item  Résoudre dans \( \mathbb{C} \) l'équation \( P(z) = 0 \). \hfill\textbf{(0,5pt)}
    \item On pose \( z_A = 1 \), \( z_B = i \), et \( z_C = \frac{1}{2}+\frac{1}{2}i \).
    \begin{enumerate}[label=\alph*)]
        \item Placer les points $A$, $B$ et $C$ d'affixes respectives $z_A$,$z_B$ et $z_C$ dans le repère orthonormal \( (O; \vec{u}, \vec{v}) \) en
        choisissant comme unité graphique 4 cm.\hfill\textbf{(0,75pt)}
        \item Démontrer que \( C \) est le milieu de \( [AB] \), puis que $C$ appartient à l'ensemble $(E)$. \hfill\textbf{(0,5pt)}
        \item  Déterminer l'affixe $z_G$ du point \( G \) barycentre du système \( \{(A,1), (B,1), (O,2)\} \). 
    \end{enumerate}
    \item  Déterminer puis construire l'ensemble $(E)$ des points $M$ du plan tels que\\ $||\overrightarrow{MA} + \overrightarrow{MA} + 2\overrightarrow{MO}||
 < 2$ \hfill\textbf{(0,5pt)}
    \item  Le jeune agriculteur atteindra-t-il son objectif ? \hfill\textbf{(0,25pt)}
    
\end{enumerate}

\section*{\textcolor{red}{\underline{EXERCICE 3} (4 points) }}
\[
\text{On considère la suite numérique } (u_n)_{n \in \mathbb{N}} \text{ définie par :}
\begin{cases}
U_{0}=6\\
U_{n+1}=\frac{1}{U_{n}}+\frac{3}{4}U_{n}, n\in\mathbb{N}
\end{cases}
\]
\begin{enumerate}
\item[1)] Calculer \( u_1 \) et \( u_2 \).\hfill\textbf{(0.5pt)}

\item[2)] Démontrer par récurrence que: \( \forall n\in\mathbb{N},\quad u_n \geq \sqrt{3} \).\hfill\textbf{(01pt)}

\item[3)] Soit $f$ la fonction définie sur $]0, +\infty[$ par \( f(x) = \frac{1}{x} + \frac{3}{4}x \).
\begin{enumerate}
\item[a)] Etudier le sens de variations de $f$.\hfill\textbf{(01pt)}
\item[b)] En déduire par récurrence que $(u_n)_{n \in \mathbb{N}}$ est strictement décroissante.\hfill\textbf{(0.5pt)}
\end{enumerate}

\item[4)] Montrer que $(u_n)_{n \in \mathbb{N}}$ est convergente et déterminer sa limite.\hfill\textbf{(01pt)}

\end{enumerate}
\section*{\textcolor{red}{\underline{PROBLÈME } (9 pts) }}
\subsection*{\textcolor{red}{\underline{Partie A }(2 pts)}}
On considère l'équation différentielle $(E):\frac{1}{2}y'+y=3e^{-2x}+2.$

\renewcommand{\labelenumi}{\theenumi)}
\begin{enumerate}[label=\arabic*)]
    \item Résoudre l'équation différentielle $(E'):\frac{1}{2}y'+y=0$.\hfill \textbf{(0,25pt)}
    \item Soit $h$ une fonction définie sur $\mathbb{R}$ par $h(x)=axe^{-2x}+b$ où $a$ et $b$ sont  des réels.\\
        Déterminer $a$ et $b$ pour que $h$ soit une solution de $(E).$\hfill \textbf{(0,5pt)}
    \item
    \begin{enumerate}[label=\alph*)]
        \item Soit $g$ une fonction dérivable sur $\mathbb{R}.$ Posons $a=6$ et $b=2.$
        Démontrer que $g$ est solution de $(E)$ si et seulement si $g-h$ est solution  de $(E').$ \hfill \textbf{(0,5pt)}
        \item En déduire l'ensemble des solutions de $(E)$.\hfill \textbf{(0,5pt)}
    \end{enumerate}
    \item Déterminer la solution $k$ de $(E)$ dont la courbe représentative $(C_{k})$ dans un repère orthonormal $(O;\vec{i},\vec{j})$ passe par le point O.\hfill \textbf{(0,25pt)}
\end{enumerate}

\subsection*{\textcolor{red}{\underline{Partie B }(7 pts)}}
Soient $f$ la fonction définie par :
\[
f(x)=
\begin{cases}
(6x-2)e^{-2x}+2 \text{ si } x\leq 0 \\
\frac{x+\ln|1-x|}{1-x} \quad\quad\quad\quad\text{ si } x > 0
\end{cases}
\]
et $\mathscr{C}_{f}$ sa courbe représentative dans un repère orthonormé d'unité graphique $2cm$.
\begin{enumerate}
    \item Justifier que l'ensemble de définition $D_{f}$ de $f$ est égal à $\mathbb{R}\setminus\{1\}$.\hfill \textbf{(0,5pt)}
    \item Etudier les limites aux bornes de $D_{f}$ et interpréter graphiquement, si possible, les résultats obtenus.\hfill \textbf{(01,5pt)}
    \item Déterminer $\lim_{x\to -\infty}\frac{f(x)}{x}$ et interpréter graphiquement le résultat.\hfill \textbf{(0,5pt)}
    \item Etudier la continuité de $f$ en 0.\> \hfill \textbf{(0,5pt)}
    \item Etudier la dérivabilité de $f$ en 0 et interpréter géométriquement les résultats obtenus.\hfill \textbf{(01pt)}
    \item Calculer $f'(x)$ puis étudier son signe sur $]-\infty, 0[$ et sur $]0, +\infty[\setminus\{1\}$.\hfill \textbf{(0,75pt)}
    \item Dresser le tableau de variations de $f$. \hfill \textbf{(0,5pt)}
    \item Montrer sur l'intervalle $]1, 2[$ que l'équation $f(x) = 0$ admet une unique solution a et que $1,2 < \alpha < 1,3$.\hfill \textbf{(0,5pt)}
    \item Construire $\mathscr{C}_{f}$ et ses asymptotes.\hfill \textbf{(0,75pt)}
    \item Calculer en $cm^{2}$ l'aire $A(E)$ de la partie $E$ du plan comprise entre les droites d'équations $x = 2$, $x = 3$, $y = -1$ et la courbe $\mathscr{C}_{f}$ de $f$.\hfill \textbf{(0,5pt)}
\end{enumerate}
\end{document}

\documentclass[12pt]{article}
\usepackage{stmaryrd}
\usepackage{graphicx}
\usepackage[utf8]{inputenc}

\usepackage[french]{babel}
\usepackage[T1]{fontenc}
\usepackage{hyperref}
\usepackage{verbatim}

\usepackage{color, soul}

\usepackage{pgfplots}
\pgfplotsset{compat=1.15}
\usepackage{mathrsfs}

\usepackage{amsmath}
\usepackage{amsfonts}
\usepackage{amssymb}
\usepackage{tkz-tab}

\usepackage{tikz}
\usetikzlibrary{arrows, shapes.geometric, fit}


\usepackage[margin=2cm]{geometry}

\begin{document}

\begin{minipage}{0.8\textwidth}
	Pathé BA                          
\end{minipage}
\begin{minipage}{0.8\textwidth}
	BAC-S2 2024
\end{minipage}

\begin{center}
\textbf{{\underline{\textcolor{red}{Premier Groupe Correction}}}}
\end{center}
%\section*{\textcolor{red}{\underline{EXERCICE 1:} (5 points) :}}
%\documentclass[a4paper,11pt]{article}
%\usepackage{amsmath,amsfonts,amssymb}
%\usepackage{graphicx}

%\title{Épreuve de Mathématiques \\ Séries S2, S2A, S4, S5}
%\author{Baccalauréat Sénégal - Session 2023}
%\date{}

%\begin{document}

%\maketitle

%\textbf{Durée : 4 heures \hfill Coefficient : 5} \\
%Les calculatrices électroniques non imprimantes avec entrée unique par clavier sont autorisées. Les calculatrices permettant d'afficher des formules ou des tracés de courbes sont interdites.

\section*{\textcolor{red}{\underline{EXERCICE 1} (3 points) }}
1) Soit \(a\) un nombre rationnel strictement positif et \(n\) un entier naturel. Donner les limites suivantes :

\[
a) \lim_{x \to 0} \frac{\ln(x+1)}{x}\quad\quad\quad\quad b) \lim_{x \to 0} \frac{e^x - 1}{x}\quad \textbf{1,5 pt} \]


2) Donner les primitives des fonctions suivantes :

\[
 f(x) = ( \exp \circ f)' \quad\quad\quad\quad f(x) = \frac{f'(x)}{f(x)}  \quad \textbf{1,5 pt}
\]

\section*{\textcolor{red}{\underline{EXERCICE 2} (4 points) }}
Un jeune agriculteur décide de pratiquer la culture sous serre dans son champ. Il place dans un repère orthonormal \( (O; \vec{u}, \vec{v}) \) deux points \( A \) et \( B \) dont les affixes respectives \( z_A \) et \( z_B \) sont des racines du polynôme \( P \) défini par :
\[
P(z) = 2z^3 - 3(1+i)z^2 + 4iz + 1 - i
\]
Il souhaite cultiver dans l'ensemble des points \( M \) de son plan de représentation tels que :
\[
||\overrightarrow{MA} + \overrightarrow{MA} + 2\overrightarrow{MO}|| < 2
\]\\
qui contient un point du segment \( [AB] \).

1) Vérifier que \(1\) et \(i\) sont des racines de \(P\). \\
2) Déterminer le polynôme \( g(z) \) tel que \( P(z) = (z-1)(z-i)g(z) \). \\
3) Résoudre dans \( \mathbb{C} \) l'équation \( P(z) = 0 \). \\
4) Placer les points \( A \), \( B \) et \( C \) dans le repère, avec \( z_A = 1 \), \( z_B = i \), et \( z_C = \frac{1+i}{2} \). \\
5) Démontrer que \( C \) est le milieu de \( [AB] \) et qu'il appartient à l'ensemble \( P \). \\
6) Déterminer l'affixe du point \( G \), barycentre du système \( \{(A,1), (B,1), (O,2)\} \). \\
7) Déterminer et construire l'ensemble \( P \). \\
8) Le jeune agriculteur atteindra-t-il son objectif ?

\section*{\textcolor{red}{\underline{EXERCICE 3} (4 points) }}
On considère la suite numérique \( (u_n)_{n \in \mathbb{N}} \) définie par :
\[
u_{n+1} = \frac{u_n + 4}{u_n + 3}
\]
1) Calculer \( u_1 \) et \( u_2 \). \\
2) Démontrer par récurrence que \( u_n \geq \sqrt{3} \) pour tout \( n \in \mathbb{N} \). \\
3) Soit \( f(x) = \frac{1}{x} + \frac{3}{4}x \). Étudier le sens de variation de \( f \). \\
4) En déduire que la suite \( (u_n) \) est strictement décroissante. \\
5) Montrer que \( (u_n) \) est convergente et déterminer sa limite.

%\section*{\textcolor{red}{\underline{Problème} (9 points) }}
%\subsection*{Partie A}
%On considère l'équation différentielle :
%\[
%(P) : y' + y = 3e^{2x} + 2
%\]
%1) Résoudre l'équation homogène associée \( (P') : y' + y = 0 \). \\
%2) Déterminer une solution particulière de \( (P) \). \\
%3) En déduire l'ensemble des solutions de \( (P) \). \\
%4) Déterminer la solution dont la courbe passe par l'origine.

%\subsection*{Partie B}
%On considère la fonction définie par :
%\[
%f(x) = 
%\begin{cases}
%(6x-2)e^{-2x} + 2 & \text{si } x < 0 \\
%x + \ln(1-x) & \text{si } x > 0
%\end{cases}
%\]
%1) Déterminer l'ensemble de définition de \( f \). \\
%2) Étudier les limites aux bornes de \( D_f \). \\
%3) Étudier la continuité de \( f \) en 0. \\
%4) Étudier la dérivabilité de \( f \) en 0. \\
%5) Calculer \( f'(x) \) et dresser son tableau de variations. \\
%6) Résoudre \( f(x) = 0 \) sur l'intervalle \( ]1,2[ \). \\
%7) Calculer l'aire sous la courbe entre \( x=2 \) et \( x=3 \).

\end{document}
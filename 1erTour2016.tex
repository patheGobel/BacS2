\documentclass[12pt]{article}
\usepackage{stmaryrd}
\usepackage{graphicx}
\usepackage[utf8]{inputenc}

\usepackage[french]{babel}
\usepackage[T1]{fontenc}
\usepackage{hyperref}
\usepackage{verbatim}

\usepackage{color, soul}

\usepackage{pgfplots}
\pgfplotsset{compat=1.15}
\usepackage{mathrsfs}

\usepackage{amsmath}
\usepackage{amsfonts}
\usepackage{amssymb}
\usepackage{tkz-tab}

\usepackage{tikz}
\usetikzlibrary{arrows, shapes.geometric, fit}


\usepackage[margin=2cm]{geometry}

\begin{document}

\begin{minipage}{0.8\textwidth}
	Pathé BA                          
\end{minipage}
\begin{minipage}{0.8\textwidth}
	BAC 2016
\end{minipage}

\begin{center}
\textbf{{\underline{\textcolor{red}{Premier Groupe}}}}
\end{center}

\section*{\textcolor{red}{\underline{Exercice 1} (4 points) :}}
\section*{\textcolor{red}{\underline{Exercice 2} (6 points) :}}
À l’occasion de ses activités culturelles, le FOSCO d’un lycée organise un jeu pour le collectif des professeurs. Une urne contenant 4 boules rouges et une boule jaune indiscernables au toucher est placée dans la cour de l’école. Chaque professeur tire simultanément 2 boules de l’urne.

- Si les deux boules sont de même couleur, il les remet dans l’urne et procède à un second tirage successif avec remise de 2 autres boules.

- Si les deux boules sont de couleurs distinctes, il les remet toujours dans l’urne, mais dans ce cas le second tirage de 2 autres boules s’effectue successivement sans remise.

1. Calculer la probabilité des événements suivants :

A : << Le professeur tire 2 boules de même couleur au premier tirage.>> \textbf{0,25 pt}

B : << Le professeur tire deux boules de couleurs différentes au premier tirage.>>  \textbf{0,25 pt}

C : << Le professeur tire deux boules de même couleur au second tirage sachant que les boules tirées au premier tirage sont de même couleur.>>  \textbf{0,5 pt}

D : << Le professeur tire deux boules de même couleur au second tirage sachant que les boules tirées au premier tirage sont de couleurs distinctes.>> \textbf{0,5 pt}

E : << Le professeur tire 2 boules de couleurs distinctes au second tirage sachant que les boules tirées au premier tirage sont de couleurs distinctes.>> \textbf{0,5 pt}

F : << Le professeur tire 2 boules de couleurs distinctes au premier et au second tirage. >> \textbf{0,5 pt}

2. Pour le second tirage, chaque boule rouge tirée fait gagner au FOSCO 1000 F et chaque boule jaune tirée fait gagner au collectif des professeurs 1000 F.
Soit \(X\) la variable aléatoire à laquelle on associe le gain obtenu par le FOSCO.

a. Déterminer les différentes valeurs prises par \(X\) et sa loi de probabilité. \textbf{1 pt}

b. Déterminer la fonction de répartition de \(X\). \textbf{1 pt}

3. Étant donné que le collectif est composé de 50 professeurs qui ont tous joué indépendamment et dans les mêmes conditions, déterminer la probabilité des événements suivants :

G : << le FOSCO réalise un gain de 100 000 F. >> \textbf{0,5 pt}

H : << le collectif des professeurs réalise un gain de 100 000 F. >> \textbf{0,5 pt}

I : << Ni gagnant, ni perdant. >> \textbf{0,5 pt}
\end{document}
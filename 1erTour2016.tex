\documentclass[12pt]{article}
\usepackage{stmaryrd}
\usepackage{graphicx}
\usepackage[utf8]{inputenc}

\usepackage[french]{babel}
\usepackage[T1]{fontenc}
\usepackage{hyperref}
\usepackage{verbatim}

\usepackage{color, soul}

\usepackage{pgfplots}
\pgfplotsset{compat=1.15}
\usepackage{mathrsfs}

\usepackage{amsmath}
\usepackage{amsfonts}
\usepackage{amssymb}
\usepackage{tkz-tab}

\usepackage{tikz}
\usetikzlibrary{arrows, shapes.geometric, fit}


\usepackage[margin=2cm]{geometry}

\begin{document}

\begin{minipage}{0.8\textwidth}
	Pathé BA                          
\end{minipage}
\begin{minipage}{0.8\textwidth}
	BAC 2016
\end{minipage}

\begin{center}
\textbf{{\underline{\textcolor{red}{Premier Groupe}}}}
\end{center}

\section*{\textcolor{red}{\underline{Exercice 1} (4 points) :}}
\begin{enumerate}
	\item On considère l’équation $(E): z^{3}-13z^{2}+59z-87=0$ , où z est un nombre complexe.
	\begin{enumerate}
		\item Déterminer la solution réelle de $(E)$.\textbf{ 0,5pt}
		\item Résoudre dans l’ensemble des nombres complexes $\mathbb{C}$ l’équation $(E)$.\textbf{ 0,5pt}
	\end{enumerate}
	\item On pose $a = 3$, $b = 5 - 2i$ et $c = 5 + 2i$.
	
			Le plan complexe étant muni d’un repère orthonormé direct $( O, \vec{u} ,\vec{v} )$, on considère les points A, B et C d’affixes respectives a, b et c. Soit M le point d’affixe z distinct de A et de B.
	\begin{enumerate}
		\item Calculer $\frac{b-a}{c-a}$. En déduire la nature du triangle ABC. \textbf{ 0,5 + 0,5pt}
		\item On pose $Z=\frac{z-2}{z-5+2i}.$
		
		Donner une interprétation géométrique de l’argument de Z.\textbf{ 0,5pt}
		
		En déduire l’ensemble des points M d’affixe z tels que Z soit un nombre réel non nul.\textbf{ 0,5pt}
  \end{enumerate}
   \item Soit $(C)$ le cercle circonscrit au triangle ABC et $I$ le point d’affixe $2 - i$.\textbf{ 0,5pt}
   \begin{enumerate}
     \item Donner l’écriture complexe de la rotation $r$ de centre $I$ et d’angle $\frac{-\pi}{2}$.\textbf{ 0,5pt}
     \item Déterminer l’image $(C")$ de $(C)$ par $r$. Construire $(C")$.\textbf{ 0,5pt}
   \end{enumerate}
\end{enumerate}
\section*{\textcolor{red}{\underline{Exercice 2} (6 points) :}}
À l’occasion de ses activités culturelles, le FOSCO d’un lycée organise un jeu pour le collectif des professeurs. Une urne contenant 4 boules rouges et une boule jaune indiscernables au toucher est placée dans la cour de l’école. Chaque professeur tire simultanément 2 boules de l’urne.

- Si les deux boules sont de même couleur, il les remet dans l’urne et procède à un second tirage successif avec remise de 2 autres boules.

- Si les deux boules sont de couleurs distinctes, il les remet toujours dans l’urne, mais dans ce cas le second tirage de 2 autres boules s’effectue successivement sans remise.

1. Calculer la probabilité des événements suivants :

A : << Le professeur tire 2 boules de même couleur au premier tirage.>> \textbf{0,25 pt}

B : << Le professeur tire deux boules de couleurs différentes au premier tirage.>>  \textbf{0,25 pt}

C : << Le professeur tire deux boules de même couleur au second tirage sachant que les boules tirées au premier tirage sont de même couleur.>>  \textbf{0,5 pt}

D : << Le professeur tire deux boules de même couleur au second tirage sachant que les boules tirées au premier tirage sont de couleurs distinctes.>> \textbf{0,5 pt}

E : << Le professeur tire 2 boules de couleurs distinctes au second tirage sachant que les boules tirées au premier tirage sont de couleurs distinctes.>> \textbf{0,5 pt}

F : << Le professeur tire 2 boules de couleurs distinctes au premier et au second tirage. >> \textbf{0,5 pt}

2. Pour le second tirage, chaque boule rouge tirée fait gagner au FOSCO 1000 F et chaque boule jaune tirée fait gagner au collectif des professeurs 1000 F.
Soit \(X\) la variable aléatoire à laquelle on associe le gain obtenu par le FOSCO.

a. Déterminer les différentes valeurs prises par \(X\) et sa loi de probabilité. \textbf{1 pt}

b. Déterminer la fonction de répartition de \(X\). \textbf{1 pt}

3. Étant donné que le collectif est composé de 50 professeurs qui ont tous joué indépendamment et dans les mêmes conditions, déterminer la probabilité des événements suivants :

G : << le FOSCO réalise un gain de 100 000 F. >> \textbf{0,5 pt}

H : << le collectif des professeurs réalise un gain de 100 000 F. >> \textbf{0,5 pt}

I : << Ni gagnant, ni perdant. >> \textbf{0,5 pt}
\section*{\textcolor{red}{\underline{Problème (10 points):}}}

\subsection*{\centering Partie A}

Soit $g$ la fonction définie par: $g(x)=-2\ln(x+1)+\frac{x}{x+1}.$
\begin{enumerate}
\item 
\begin{enumerate}
\item[a.] Déterminer $Dg$, puis calculer les limites de g aux bornes de $Dg$. \textbf{0,75 pt}
\item[b.] Calculer $g'(x)$ , étudier son signe et dresser le tableau de variations de g. \textbf{1 pt}
\end{enumerate}
\item 
\begin{enumerate}
\item[a.] Calculer $g(0)$. Montrer que l’équation $g(x) = 0$ admet exactement deux solutions dont l’une que l’on désigne $\alpha \in ]-0,72, -0,71[.$ \textbf{0,25 +0,5 pt}
\item[b.] Déterminer le signe de $g(x)$. \textbf{0,5 pt}
\end{enumerate}
\end{enumerate}

\subsection*{\centering Partie B}
\end{document}
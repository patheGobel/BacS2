\documentclass[12pt]{article}
\usepackage{stmaryrd}
\usepackage{graphicx}
\usepackage[utf8]{inputenc}

\usepackage[french]{babel}
\usepackage[T1]{fontenc}
\usepackage{hyperref}
\usepackage{verbatim}

\usepackage{color, soul}

\usepackage{pgfplots}
\pgfplotsset{compat=1.15}
\usepackage{mathrsfs}

\usepackage{amsmath}
\usepackage{amsfonts}
\usepackage{amssymb}
\usepackage{tkz-tab}

\usepackage{tikz}
\usetikzlibrary{arrows, shapes.geometric, fit}


\usepackage[margin=2cm]{geometry}

\begin{document}

\begin{minipage}{0.8\textwidth}
	Pathé BA                          
\end{minipage}
\begin{minipage}{0.8\textwidth}
	BAC 2024
\end{minipage}

\begin{center}
\textbf{{\underline{\textcolor{red}{Premier Groupe}}}}
\end{center}
\section*{\textcolor{red}{\underline{EXERCICE 1:} (5 points) :}}

1. Dans le plan complexe muni d'un repère orthonormé, on considère les points A, B et C d'affixes respectives \(z_A = -3i\), \(z_B = -2\) et \(z_C = 1+2i\).

\begin{enumerate}
    \item[a.] Déterminer le module et un argument du quotient \(\frac{z_C - z_B}{z_A - z_B}\).
    \item[b.] En déduire la nature du triangle ABC.
    \item[c.] Déterminer l'affixe \(z_D\) du point D tel que le quadrilatère BADC soit un carré et on précisera le centre et le rayon.
    \item[d.] Montrer que les points A, B, C et D appartiennent à un même cercle dont on précisera le centre et le rayon.
\end{enumerate}

2. On considère les points \(M\) et \(M'\) d'affixes respectives \(z = x + iy\) et \(z' = x' + iy'\) où \(x, y, x', y'\) sont des réels.

Soit \(S\) l'application du plan dans le plan d'expression analytique :
\[
\begin{cases}
    x' = x - y + 2 \\
    y' = x + y - 1
\end{cases}
\]

\begin{enumerate}
    \item[a.] Montrer que l'écriture complexe de \(S\) est : \(z' = (1 + i)z + 2 - i\). (0,5 point)
    \item[b.] Déterminer la nature et les éléments caractéristiques de \(S\). (0,75 point)
    \item[c.] Déterminer l'image par \(S\) de la droite \((D)\) d'équation \(x + y + 1 = 0\). (0,5 point)
    \item[d.] Déterminer l'ensemble des points \(M\) dont l'affixe \(z\) vérifie \(\mid(1 + i)z + 2 - i \mid= 2\). (0,5 point)
\end{enumerate}
\section*{\textcolor{red}{\underline{EXERCICE 2:} (5 points) :}}

Une entreprise fabrique des articles dans deux unités de production notées \( U_1 \) et \( U_2 \). L’unité \( U_1 \) assure 60\% de la production.

On a constaté que :

- 3\% des articles provenant de l’unité \( U_1 \) présentent un défaut de fabrication.

- 8\% des articles provenant de l’unité \( U_2 \) présentent un défaut de fabrication.

L’entreprise envisage de mettre en place un test de contrôle de ces articles avant leur mise en vente. Ce contrôle détecte et élimine 82\% des articles défectueux, mais il élimine également à tort 4\% des articles non défectueux. Les articles non éliminés sont alors mis en vente.

L’entreprise souhaite que moins de 1\% des articles vendus soient défectueux.

À l’aide des informations ci-dessus et des outils mathématiques au programme :
\begin{enumerate}
    \item Démontrer que 5\% des articles produits présentent un défaut de fabrication. \textbf{(02 points)}
    \item En prenant au hasard un article fabriqué, montrer que la chance que cet article soit mis en vente après contrôle est de 0,921. \textbf{(02 points)}
    \item Vérifier si ce contrôle permet à l’entreprise de réaliser son souhait.\hfill \textbf{(01 point)}
\end{enumerate}
\section*{\textcolor{red}{\underline{PROBLEME :} ( 10 points ).}}
\subsection*{ \underline{PARTIE A } ( 2 points ) }
\begin{enumerate}
\item Pour tout $x < 0$, on pose : $u(x)=x+1-e^{-x}$.

Etudier le signe de $1-e^{-x}$ pour $x < 0$.

En déduire que pour tout $x < 0, u(x) < 0 $.

\item Pour tout $x > 0,$ on pose : $v(x)=x-1-\ln x$.
\begin{enumerate}
\item[a.] Dresser le tableau de variations de $v$.
\item[b.] En déduire le signe de $v(x)$ pour $x > 0 $.
\end{enumerate}
\end{enumerate}
\subsection*{ \underline{PARTIE B } ( 8 points ) }
\[
\text{Soient f la fonction définie par} 
f(x)=
\begin{cases}
xe^{x}-x-1 \quad\quad  \text{ si }  x \leq 0\\
x^{2}-1-2x\ln x \quad \text{ si } x > 0 
\end{cases}
\]
et $C_{f}$ sa courbe représentative dans un repère orthonormé (O;$\vec{i}$;$\vec{j}$) d'unité 1 cm.
\begin{enumerate}
\item
\begin{enumerate}
\item[a.] Montrer que l'ensemble de définition de $f$ est $\mathbb{R}$.\textbf{ 0,5 points}
\item[b.] Etudier les limites de en $-\infty$ et en $+\infty$.\textbf{ 0,5 points}
\item[c.]Montrer que la droite $(D)$ d'équation $y=-x-1$ est asymptote à $C_{f}$ en $-\infty$. \textbf{ 0,25 points}
Préciser la position de $(C_{f})$ par rapport à $(D)$ sur $]-\infty, 0[.$ \textbf{ 0,25 points}
\item[d.]Etudier la nature de la branche infinie de $(C_{f})$ en $+\infty$.\textbf{ 0,5 points}
\end{enumerate}
\item
\begin{enumerate}
\item[a.]Etudier la contuinité de $f$ en  $0$. \textbf{ 0,5 point}
\item[b.]Etudier la dérivabilité de $f$ en $0$. \textbf{ 0,5 point}

	Interpreter  graphiquement les résultats obtenus. \textbf{ 0,5 point}

\end{enumerate}
\item
\begin{enumerate}
\item[a.] Montrer que pour tout $x < 0 $, $f'(x)=u(x)e^{x}$. \textbf{ 0,5 point}

	En déduire le signe de $f'(x)$ sur $]-\infty, 0[.$ \textbf{ 0,25 point}
\item[b.] Montrer que pour tout $ x > 0$, $f'(x)=2v(x).$ \textbf{ 0,5 point}

	En déduire le signe de $f'(x)$ sur $]0, +\infty[.$ \textbf{ 0,25 point}
\item[c.] Dresser le tableau de variations de $f$. \textbf{ 0,5 point}
\end{enumerate}
\item Tracer $(D)$ et $(C_{f})$ dans le plan muni du repère (O,$\vec{i}$,$\vec{j}$). \textbf{ 0,25 point + 0,5 point}

\item Soit $g$ la restriction de $f$ à l'intervalle $]0, +\infty[.$
\begin{enumerate}
\item[a.] Montrer que $g$ admet une bijection réciproque $g^{-1}$ dont on précisera l'ensemble de définition et le sens de variations. \textbf{ 0,25 point + 0,25 point + 0,25 point}
\item[b.] Tracer la courbe representative $C_{g^{-1}}$ de $g^{-1}$ dans le plan muni du repère \\
			(O,$\vec{i}$,$\vec{j}$). \textbf{ 0,25 point}
\end{enumerate}
\item Soit $\lambda$ un réel strictement négatif.
\begin{enumerate}
\item[a.] Exprimer l'aire $A(\lambda)$ en fonction de $\lambda$ la partie du plan délimitée par les droites d'équations $x=\lambda$, $x=0$, $y=-x-1$ et la courbe $(C_{f})$. \textbf{ 0,25 point}
\item[b.] En déduire $\lim_{\lambda \to -\infty}A(\lambda)$. \textbf{ 0,25 point}
\end{enumerate}
\end{enumerate}
\end{document}
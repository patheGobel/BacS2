\documentclass[12pt]{article}
\usepackage{stmaryrd}
\usepackage{graphicx}
\usepackage[utf8]{inputenc}

\usepackage[french]{babel}
\usepackage[T1]{fontenc}
\usepackage{hyperref}
\usepackage{verbatim}

\usepackage{color, soul}

\usepackage{pgfplots}
\pgfplotsset{compat=1.15}
\usepackage{mathrsfs}

\usepackage{amsmath}
\usepackage{amsfonts}
\usepackage{amssymb}
\usepackage{tkz-tab}

\usepackage{tikz}
\usetikzlibrary{arrows, shapes.geometric, fit}


\usepackage[margin=2cm]{geometry}

\begin{document}

\begin{minipage}{0.8\textwidth}
	Pathé BA                          
\end{minipage}
\begin{minipage}{0.8\textwidth}
	BAC 2024
\end{minipage}

\begin{center}
\textbf{{\underline{\textcolor{red}{Premier Groupe Correction}}}}
\end{center}
\section*{\textcolor{red}{\underline{EXERCICE 1:} (5 points) :}}

1. Dans le plan complexe muni d'un repère orthonormé, on considère les points A, B et C d'affixes respectives \(z_A = -3i\), \(z_B = -2\) et \(z_C = 1+2i\).

\begin{enumerate}
    \item[a.] Déterminer le module et un argument du quotient \(\frac{z_C - z_B}{z_A - z_B}\).
    \item[b.] En déduire la nature du triangle ABC.
    \item[c.] Déterminer l'affixe \(z_D\) du point D tel que le quadrilatère BADC soit un carré et on précisera le centre et le rayon.
    \item[d.] Montrer que les points A, B, C et D appartiennent à un même cercle dont on précisera le centre et le rayon.
\end{enumerate}

2. On considère les points \(M\) et \(M'\) d'affixes respectives \(z = x + iy\) et \(z' = x' + iy'\) où \(x, y, x', y'\) sont des réels.

Soit \(S\) l'application du plan dans le plan d'expression analytique :
\[
\begin{cases}
    x' = x - y + 2 \\
    y' = x + y - 1
\end{cases}
\]

\begin{enumerate}
    \item[a.] Montrer que l'écriture complexe de \(S\) est : \(z' = (1 + i)z + 2 - i\). (0,5 point)
    \item[b.] Déterminer la nature et les éléments caractéristiques de \(S\). (0,75 point)
    \item[c.] Déterminer l'image par \(S\) de la droite \((D)\) d'équation \(x + y + 1 = 0\). (0,5 point)
    \item[d.] Déterminer l'ensemble des points \(M\) dont l'affixe \(z\) vérifie \(\mid(1 + i)z + 2 - i \mid= 2\). (0,5 point)
\end{enumerate}
\section*{\textcolor{red}{\underline{EXERCICE 2:} (5 points) :}}

Une entreprise fabrique des articles dans deux unités de production notées \( U_1 \) et \( U_2 \). L’unité \( U_1 \) assure 60\% de la production.

On a constaté que :

- 3\% des articles provenant de l’unité \( U_1 \) présentent un défaut de fabrication.

- 8\% des articles provenant de l’unité \( U_2 \) présentent un défaut de fabrication.

L’entreprise envisage de mettre en place un test de contrôle de ces articles avant leur mise en vente. Ce contrôle détecte et élimine 82\% des articles défectueux, mais il élimine également à tort 4\% des articles non défectueux. Les articles non éliminés sont alors mis en vente.

L’entreprise souhaite que moins de 1\% des articles vendus soient défectueux.

À l’aide des informations ci-dessus et des outils mathématiques au programme :
\begin{enumerate}
    \item Démontrer que 5\% des articles produits présentent un défaut de fabrication. \textbf{(02 points)}
    \item En prenant au hasard un article fabriqué, montrer que la chance que cet article soit mis en vente après contrôle est de 0,921. \textbf{(02 points)}
    \item Vérifier si ce contrôle permet à l’entreprise de réaliser son souhait. \textbf{(01 point)}
\end{enumerate}

\end{document}
\documentclass[12pt]{article}
\usepackage{stmaryrd}
\usepackage{graphicx}
\usepackage[utf8]{inputenc}

\usepackage[french]{babel}
\usepackage[T1]{fontenc}
\usepackage{hyperref}
\usepackage{verbatim}

\usepackage{color, soul}

\usepackage{pgfplots}
\pgfplotsset{compat=1.15}
\usepackage{mathrsfs}

\usepackage{amsmath}
\usepackage{amsfonts}
\usepackage{amssymb}
\usepackage{tkz-tab}

\usepackage{tikz}
\usetikzlibrary{arrows, shapes.geometric, fit}


\usepackage[margin=2cm]{geometry}

\begin{document}

\begin{minipage}{0.8\textwidth}
	Pathé BA                          
\end{minipage}
\begin{minipage}{0.8\textwidth}
	BAC 2016
\end{minipage}

\begin{center}
\textbf{{\underline{\textcolor{red}{Premier Groupe Correction}}}}
\end{center}
\section*{\textcolor{green}{\underline{Correction Exercice 1} (6 points) :}}
\begin{enumerate}
	\item On considère l’équation $(E): z^{3}-13z^{2}+59z-87=0$ , où z est un nombre complexe.
	\begin{enumerate}
		\item Déterminons la solution réelle de $(E)$.\textbf{ 0,5pt}
		
		Si la solution est réelle alors elle est de la forme $z_{0}=a$ où $a\in \mathbb{R}$
		
		donc $a^{3}-13a^{2}+59a-87=0$ en cherchant parmis les solutions évidentes, on trouve que $a=3$
		
		\item Résolvons dans l’ensemble des nombres complexes $\mathbb{C}$ l’équation $(E)$.\textbf{ 0,5pt}
		
		Par Honer:\\
		\begin{tabular}{|c|c|c|c|c|}
					\hline
					& 1 & -13 & 59 & -87\\
					\hline
					3 &  & 3 & 30 & 87\\
					\hline
					& 1 & 10 & 29 & 0\\
					\hline
		\end{tabular}
		
		Donc $z^{3}-13z^{2}+59z-87=(z-3)(z^{2}+10z+29)=0$
		
		Résolvons $z^{2}+10z+29=0$
		
		$\Delta' = 25-29=-4\implies \Delta'=(2i)^{2}$ ou $\Delta'=(-2i)^{2}$
		
		$z_{1}=-5-2i$, $z_{2}=-5+2i$
		
		$z_{1}=-5-2i$, $z_{2}=-5+2i$
		
		\textcolor{green}{\boxed{S=\left\lbrace 3, -5-2i, -5+2i \right\rbrace }}
	\end{enumerate}
	\item On pose $a = 3$, $b = 5 - 2i$ et $c = 5 + 2i$.
	
			Le plan complexe étant muni d’un repère orthonormé direct $( O, \vec{u} ,\vec{v} )$, on considère les points A, B et C d’affixes respectives a, b et c. Soit M le point d’affixe z distinct de A et de B.
	\begin{enumerate}
		\item Calculons $\frac{b-a}{c-a}$.
		
		$\frac{b-a}{c-a}=\frac{2 - 2i}{2 + 2i}=\frac{(2 - 2i)(2 - 2i)}{4}=-2i$
		
		Déduisons la nature du triangle ABC. \textbf{ 0,5 + 0,5pt}
		
		$\arg\left( \frac{b-a}{c-a} \right) =-\frac{\pi}{2}[2\pi] $
		
		Comme $\arg\left( \frac{b-a}{c-a} \right) =(\overrightarrow{AC},\overrightarrow{AB})=-\frac{\pi}{2}[2\pi] $ et $\mid\frac{b-a}{c-a}\mid=2$ ( c'est-à-dire AB=2AC ) donc:
		
		ABC est un triangle rectangle en A.
		
		\begin{align*}\textbf{\textcolor{red}{Rappel:}}\\
				\arg\left( \frac{b-a}{c-a} \right) &=\arg(b-a)-\arg(c-a)\\
				&=(\vec{u},\overrightarrow{AB})-(\vec{u},\overrightarrow{AC})\\
				&=(\vec{u},\overrightarrow{AB})+(\overrightarrow{AC},\vec{u})\\
				&=(\overrightarrow{AC},\vec{u})+(\vec{u},\overrightarrow{AB})\\
				&=(\overrightarrow{AC},\overrightarrow{AB})
		\end{align*}

		
		\item On pose $Z=\frac{z-2}{z-5+2i}.$
		
		Donnons une interprétation géométrique de l’argument de Z.\textbf{ 0,5pt}
		
		$Z=\frac{z-2}{z-5+2i}=\frac{z-(2)}{z-(5-2i)}=\frac{z_{M}-z_{A}}{z_{M}-z_{B}}$
		
		$ \arg(Z)=\arg(\frac{z-z_{A}}{z-z_{B}})=(\overrightarrow{BM},\overrightarrow{AM})=-(\overrightarrow{AM},\overrightarrow{BM})=(\overrightarrow{MA},\overrightarrow{MB})$
		
		\textcolor{green}{L'argument de Z représente l'angle orienté entre les segments $\overrightarrow{BM}$ et $\overrightarrow{AM}$.}
		
		Déduisons-en l’ensemble des points M d’affixe z tels que Z soit un nombre réel non nul.\textbf{ 0,5pt}
		
		Si Z est un nombre réel non nul alors \\$ \arg(Z)=0\implies \arg(\frac{z-z_{A}}{z-z_{B}})=(\overrightarrow{BM},\overrightarrow{AM})=(\overrightarrow{MA},\overrightarrow{MB})=0[\pi]$.
		
		\textcolor{green}{Donc l’ensemble des points M est la droite (AB) privé du segment [AB]}
  \end{enumerate}
   \item Soit $(C)$ le cercle circonscrit au triangle ABC et $I$ le point d’affixe $2 - i$.\textbf{ 0,5pt}
   \begin{enumerate}
     \item Donnons l’écriture complexe de la rotation $r$ de centre $I$ et d’angle $\frac{-\pi}{2}$.\textbf{ 0,5pt}
     Puisque c'est une rotation donc elle est de la forme $z'=az+b$ ou $a=e^{\frac{-\pi}{2}}$ et $w=I=\frac{b}{1-a}$
     
     Donc $b=(2 - i)\times (1-e^{\frac{-\pi}{2}})=(2 - i)\times (1+i)=(2 +2i-i+1)=3+i$
     
     Finalement, l’écriture complexe de la rotation $r$ :
     \textcolor{green}{\boxed{z'=e^{\frac{-\pi}{2}}z+3+i}}
     \item Déterminons l’image $(C')$ de $(C)$ par $r$.
     
     Cherchons les élémenents caractéristiques de $(C')$
     
     Comme $(C')$ est l'image de $(C)$ par une rotation, donc ils ont les mêmes rayons. 
     
     C'est-à-dire: $r=\frac{BC}{2}=|\frac{5+2i-5+2i}{2}|=2$
     
     son centre:
     
     $r(z)=e^{\frac{-\pi}{2}}z+3+i$
     
     $r(z_{I})=z_{I'}=-i(5)+3+i=3-4i$
     
     \textcolor{green}{\boxed{z_{I'}=3-4i}}
     
     Finalement, $C'(I',2)$
     
     Construisons $(C')$.\textbf{ 0,5pt}
\newpage   
\begin{figure}[h]
\centering
\includegraphics[width=0.8\textwidth]{CerlceEtImage.png}
\caption{C' image de C par rotation de centre de I(2-i) et d'angle -$\frac{\pi}{2}$}
\label{fig:monimage}
\end{figure}
   \end{enumerate}
\end{enumerate}
\section*{\textcolor{green}{\underline{Correction Exercice 2} (6 points) :}}
À l’occasion de ses activités culturelles, le FOSCO d’un lycée organise un jeu pour le collectif des professeurs. Une urne contenant 4 boules rouges et une boule jaune indiscernables au toucher est placée dans la cour de l’école. Chaque professeur tire simultanément 2 boules de l’urne.

- Si les deux boules sont de même couleur, il les remet dans l’urne et procède à un second tirage successif avec remise de 2 autres boules.

- Si les deux boules sont de couleurs distinctes, il les remet toujours dans l’urne, mais dans ce cas le second tirage de 2 autres boules s’effectue successivement sans remise.

\textcolor{green}{1. Calculons la probabilité des événements suivants :}

\begin{itemize}
\item \underline{\textcolor{green}{Pour l'évènement A:}}

$P(A)=\frac{card(A)}{card(\Omega)}$

\textcolor{green}{Cherchons $card(\Omega)$ et card(A)}
	\begin{itemize}
	\item \underline{Pour $card(\Omega)$}
	\begin{align*}
	card(\Omega)&=C_{5}^{2}\\
			&=10
	\end{align*}
	
	\begin{center}
		\textcolor{blue}{\boxed{card(\Omega)=10}}
	\end{center}
	\item \underline{Pour $card(A)$}
	
		A : << Le professeur tire 2 boules de même couleur au premier tirage.>>

		\textcolor{green}{Autrement dit}: A : << RR >>. C'est-à-dire tirer 2 rouge parmis les 4 rouges.

		Donc card(A)=$C_{4}^{2}$

		\begin{align*}
			card(A)&=C_{4}^{2}\\
					&=6
		\end{align*}

		\begin{center}
			\textcolor{blue}{\boxed{card(A)=6}}
		\end{center}
	\end{itemize}
	Ainsi, la probabilité de l'évènement A est:

	$P(A)=\frac{card(A)}{card(\Omega)}=\frac{6}{10}$
		\begin{center}
			\textcolor{green}{\boxed{P(A)=\frac{3}{5}}}
		\end{center}
\item \underline{\textcolor{green}{Pour l'évènement B:}}

B : << Le professeur tire deux boules de couleurs différentes au premier tirage.>>  \textbf{0,25 pt}
			\textcolor{green}{Autrement dit}: B : << RJ >>. C'est-à-dire tirer 1 rouge parmis les 4 rouges et 1 jaune .

		\begin{align*}
			card(B)&=C_{4}^{1}\times C_{1}^{1}\\
					&=4
		\end{align*}

		\begin{center}
			\textcolor{blue}{\boxed{card(B)=4}}
		\end{center}

	Ainsi, la probabilité de l'évènement B est:

	$P(B)=\frac{card(B)}{card(\Omega)}=\frac{4}{10}$
		\begin{center}
			\textcolor{green}{\boxed{P(B)=\frac{2}{5}}}
		\end{center}
\end{itemize}

\textcolor{green}{En remarquant que les évènements A et B sont des évènements contraires, et que les évènements C, D, E et F sont dépendants soit de la réalisation de A ou de B, contruisons un arbe de probabilité.}

\begin{tikzpicture}[level distance=3cm,
  level 1/.style={sibling distance=6cm},%Ecarte les branches des 1eme ramifications
  level 2/.style={sibling distance=2.5cm},%Ecarte les branches des  2eme ramifications
  %level 3/.style={sibling distance=2cm}]%Ecarte les branches des 3eme ramifications
    every node/.style={text width=2cm, align=center}]%Permet de spécifier une largeur pour chaque nœud
  \node {}
    child {node {$B$}
      child {node {$RR$}    
      }
      child {node {$JR$}    
      }
    }      
    child {node {$A$} 
      child {node {$JR$}    
      }
      child {node {$JJ$}    
      } 
      child {node {$RR$}    
      }
    };
\node at (-4,-1.5) [right] {$\frac{2}{5}$};
\node at (1.3,-1.5) [right] {$\frac{3}{5}$};

\node at (-5,-4) [right] {$\frac{12}{20}$};
\node at (-3,-4) [right] {$\frac{8}{20}$};

\node at (-0.1,-4) [right] {$\frac{8}{25}$};
\node at (1.5,-5.2) [right] {$\frac{1}{25}$};
\node at (3.9,-4.5) [right] {$\frac{16}{25}$};

\end{tikzpicture}

\begin{itemize}
\item \underline{\textcolor{green}{Pour P(C):Tirage successif avec remise de 2 autres boules}}

C : << Le professeur tire deux boules de même couleur au second tirage sachant que les boules tirées au premier tirage sont de même couleur.>>  \textbf{0,5 pt}

C : << RR ou JJ >>

P(C)=P(RR ou JJ)=$P_{A}({RR-ou-JJ})=P(RR)+P(JJ)=\frac{16}{25}+\frac{1}{25}=\frac{17}{25}$

		\begin{center}
			\textcolor{green}{\boxed{P(C)=\frac{17}{25}}}
		\end{center}

\item \underline{\textcolor{green}{Pour P(D):tirage successif sans remise de 2 autres boules}}

D : << Le professeur tire deux boules de même couleur au second tirage sachant que les boules tirées au premier tirage sont de couleurs distinctes.>> \textbf{0,5 pt}

D : << RJ-JR>> 

\begin{tikzpicture}
  % Case 1
  \draw (0,0) rectangle (1.5,1.5);
  \node at (0.5,1.9) {R};
  \node at (0.8,0.8) {4};
  \node at (3,0.8) {X};
  
  % Case 2
  \draw (5,0) rectangle (6.5,1.5);
  \node at (5.7,1.8) {J};
  \node at (5.7,0.8) {1};
  \node at (8,0.8) {X};
  \node at (9,0.8) {2};
\end{tikzpicture}\\ 

P(D)=P(RJ)=$P_{B}({RJ})=P(RJ)=\frac{4\times 2}{20}=\frac{2}{5}$

		\begin{center}
			\textcolor{green}{\boxed{P(D)=\frac{2}{5}}}
		\end{center}

\item \underline{\textcolor{green}{Pour P(E):}}

E : << Le professeur tire 2 boules de couleurs distinctes au second tirage sachant que les boules tirées au premier tirage sont de couleurs distinctes.>> \textbf{0,5 pt}

E : << RR >>

P(E)=P(RR)=$P_{B}({RR})=P(RR)=\frac{12}{20}=\frac{3}{5}$

		\begin{center}
			\textcolor{green}{\boxed{P(D)=\frac{3}{5}}}
		\end{center}

\item \underline{\textcolor{green}{Pour P(F):}}

F : << Le professeur tire 2 boules de couleurs distinctes au premier et au second tirage. >> \textbf{0,5 pt}

E : << JR >>

P(F)=P(JR)=$P(A\cap F)=\frac{2}{5}\times \frac{2}{5}$

		\begin{center}
			\textcolor{green}{\boxed{P(A\cap F)=\frac{4}{25}}}
		\end{center}
\end{itemize}
2. Pour le second tirage, chaque boule rouge tirée fait gagner au FOSCO 1000 F et chaque boule jaune tirée fait gagner au collectif des professeurs 1000 F.
Soit \(X\) la variable aléatoire à laquelle on associe le gain obtenu par le FOSCO.

a. Déterminer les différentes valeurs prises par \(X\) et sa loi de probabilité. \textbf{1 pt}

b. Déterminer la fonction de répartition de \(X\). \textbf{1 pt}

3. Étant donné que le collectif est composé de 50 professeurs qui ont tous joué indépendamment et dans les mêmes conditions, déterminer la probabilité des événements suivants :

G : << le FOSCO réalise un gain de 100 000 F. >> \textbf{0,5 pt}

H : << le collectif des professeurs réalise un gain de 100 000 F. >> \textbf{0,5 pt}

I : << Ni gagnant, ni perdant. >> \textbf{0,5 pt}
\section*{\textcolor{red}{\underline{Problème (10 points):}}}

\subsection*{\centering Partie A}

Soit $g$ la fonction définie par: $g(x)=-2\ln(x+1)+\frac{x}{x+1}.$
\begin{enumerate}
\item 
\begin{enumerate}
\item[a.] Déterminons $Dg$, puis calculons les limites de g aux bornes de $Dg$. \textbf{0,75 pt}

\textcolor{green}{\underline{$Dg$:}}

$g$ existe ssi $ x+1 >0 $ et $x+1\neq 0$ $\implies$ $ x>-1 $ et $x\neq -1$ $\implies$ $x\in]-1,+\infty[$ et $x\neq -1$ 
		\begin{center}
			\textcolor{green}{\boxed{Df=]-1,+\infty[}}
		\end{center}
\textcolor{green}{\underline{Les limites de g aux bornes de $Dg$:}}

Les bornes de $Dg$ sont $-1$ et $+\infty$

\textcolor{green}{\underline{En -1}:}

\[
\lim_{x \to -1^{+}}g(x)=\lim_{x \to -1^{+}}-2\ln(x+1)+\frac{x}{x+1}:
\begin{cases}
\lim_{x \to -1^{+}}-2\ln(x+1)=+\infty\\
\lim_{x \to -1^{+}}\frac{x}{x+1}=-\infty
\end{cases}
\text{Par somme F.I}
\]
Levons l'indétermination
\begin{align*}
\lim_{x \to -1^{+}}g(x)&=\lim_{x \to -1^{+}}-2\ln(x+1)+\frac{x}{x+1}\\
&=\lim_{x \to -1^{+}}\frac{-2(x+1)\ln(x+1)+x}{x+1}:
\begin{cases}
\lim_{x \to -1^{+}}-2(x+1)\ln(x+1)+x=-1\\
\lim_{x \to -1^{+}}x+1=0^{+}
\end{cases}
\end{align*}
\[\text{Par quotient }\lim_{x \to -1^{+}}g(x)=-\infty\]
		\begin{center}
			\textcolor{green}{\boxed{\lim_{x \to -1^{+}}g(x)=-\infty}}
		\end{center}		
\textcolor{green}{\underline{En $+\infty$}:}
\[
\lim_{x \to -1^{+}}g(x)=\lim_{x \to +\infty}-2\ln(x+1)+\frac{x}{x+1}:
\begin{cases}
\lim_{x \to +\infty}-2\ln(x+1)=-\infty\\
\lim_{x \to +\infty}\frac{x}{x+1}=1
\end{cases}
\text{Par somme:}
\]
		\begin{center}
			\textcolor{green}{\boxed{\lim_{x \to +\infty}g(x)=-\infty}}
		\end{center}
\item[b.] Calculons $g'(x)$ 

$x \mapsto -2\ln(x+1)$ est dérivable sur $ Dg $   et $x\mapsto \frac{x}{x+1}$ est dérivable sur $ Dg $

Par somme, g est la somme de deux fonctions dérivables sur Dg donc dérivable.
\begin{align*}
g'(x)&=-2\frac{1}{x+1}+\frac{x+1-x}{(x+1)^{2}}\\
			&=\frac{-2}{x+1}+\frac{1}{(x+1)^{2}}\\
			&=\frac{-2x-2+1}{(x+1)^{2}}\\
			&=\frac{-2x-1}{(x+1)^{2}}
\end{align*}
		\begin{center}
			\textcolor{green}{\boxed{g'(x)=\frac{-2x-1}{(x+1)^{2}}}}
		\end{center}
Etudions le de $g'(x)$

Le signe de $g'(x)$ dépend de numérateur.

Or $\forall x \in]-1,-\frac{1}{2}[$, $g'(x)>0$ donc $g$ est croissante

$\forall x \in ]-\frac{1}{2},+\infty[$, $g'(x)<0$ donc $g$ est décroissante

Dressons le tableau de variations de g. \textbf{1 pt}

\begin{tikzpicture}
 \tkzTabInit{$x$/1,$g'$/1,$g$/2}{$-1$,$-\frac{1}{2}$,$+\infty$}
 \tkzTabLine{,+,z,-,}
 \tkzTabVar{-/$-\infty$,+/$0.39$,-/$-\infty$}
 \tkzTabVal{2}{3}{0.5}{$0$}{0}
 \tkzTabVal{1}{2}{0.4}{$\alpha$}{0}
 % Add a movable label
 %\node at (3.5, -1.5) {$7$};
\end{tikzpicture}
\end{enumerate}
\item 
\begin{enumerate}
\item[a.] Calculons $g(0)$.

$g(0)=-2\ln(0+1)+\frac{0}{0+1}=0$

\textcolor{green}{g(0)=0}

Montrons que l’équation $g(x) = 0$ admet exactement deux solutions dont l’une que l’on désigne $\alpha \in ]-0,72, -0,71[.$ \textbf{0,25 +0,5 pt}

\textcolor{green}{\underline{Existant de la première solution:}}

Comme \textcolor{green}{g(0)=0} donc 0 est solution de l'équation $g(x) = 0$

\textcolor{green}{\underline{Existant de la deuxiéme solution:}}

Les solutions existes ssi $g(-0,72)\times g(-0,71)<0$

$g(-0,72)=-0,025$ et $g(-0,71)=0,027$ donc $g(-0,72)\times g(-0,71)<0$

Donc il existe une solution $\alpha \in ]-0,72, -0,71[$

\textcolor{green}{\underline{inicité:}}

Comme $g$ est continue et strictement coissant sur $]-0,72, -0,71[$ donc la solution est unique.

\item[b.] Déterminons le signe de $g(x)$. \textbf{0,5 pt}

D'après le tableau de variation, 

\textcolor{green}{$\forall x \in \left( ]-1;\alpha[\cup]0;+\infty[\right),\quad g(x)<0$}

\textcolor{green}{$\forall x \in \left( ]\alpha;0[\right),\quad g(x)>0$}
\end{enumerate}
\end{enumerate}

\subsection*{\centering Partie B}
$$
 \text{ Soit f la fonction définie par } 
\begin{cases}
f(x)=\frac{x^{2}}{\ln(x+1)}\quad\quad\quad\quad \text{ si } x>-1\\
f(x)=(1+x)e^{-x-1} \quad\text{ si } x\leq -1\\
f(0)=0
\end{cases}
$$
\begin{enumerate}
\item
\begin{enumerate}
\item[a.] Montrons que $D_{f}=\mathbb{R}$
$$
 \text{ Posons } 
\begin{cases}
f(x)=f_{1}(x)=\frac{x^{2}}{\ln(x+1)}\quad\quad\quad\quad \text{ si } x>-1\\
f(x)=f_{2}(x)=(1+x)e^{-x-1} \quad\text{ si } x\leq -1\\
f(0)=0
\end{cases}
$$
-$f_{1}$ existe ssi $x+1>0$ et $x>-1$ donc

($x+1>0$ et $x>-1$) $ \implies x>-1\implies x \in \left]-1;+\infty \right[$

Donc \textcolor{green}{$Df_{1}=\left]-1;+\infty \right[$}

-\textcolor{green}{$Df_{1}=\left]-\infty; -1\right]$}

Donc $Df=\left( \left]-1;+\infty \right[ \right)  \cap \left( \left]-\infty; -1\right] \right) =\mathbb{R} $

\textcolor{green}{\boxed{Df=\mathbb{R}}}

Calculons les limites aux bornes de $D_{f}$. $\textbf{0,75 pt}$

Les bornes de Df sont $-\infty$ et $+\infty$
\begin{itemize}
\item \underline{\textcolor{green}{En $-\infty$}}
$$
\lim_{x \to -\infty}f(x)=\lim_{x \to -\infty}(1+x)e^{-x-1}=-\infty
$$
\begin{center}
\textcolor{green}{\boxed{\lim_{x \to -\infty}f(x)=-\infty}}
\end{center}

\item \underline{\textcolor{green}{En $+\infty$}}

\begin{align*}
\lim_{x \to +\infty}f(x)&=\lim_{x \to +\infty}\frac{x^{2}}{\ln(x+1)}\times \frac{1}{\frac{x+1}{x+1}}\\
&=\lim_{x \to +\infty}\frac{x^{2}}{\frac{\ln(x+1)(x+1)}{x+1}}\\
&=\lim_{x \to +\infty}\frac{1}{\frac{\ln(x+1)}{x+1}}\times\frac{x^{2}}{x+1}:
\begin{cases}
\lim_{x \to +\infty}\frac{1}{\frac{\ln(x+1)}{x+1}}=\frac{1}{0^{+}}=+\infty\\
\lim_{x \to +\infty}\frac{x^{2}}{x+1}=+\infty
\end{cases}
\end{align*}

\begin{center}
\textcolor{green}{\boxed{\lim_{x \to +\infty}f(x)=+\infty}}
\end{center}

\end{itemize}

\item[b.] Etudions la nature des branches infinies. $\textbf{0,75 pt}$
\begin{itemize}
\item \underline{\textcolor{green}{En $-\infty$}}
\[\text{Comme} \lim_{x \to -\infty}f(x)=-\infty, \text{calculons } \lim_{x \to -\infty}\frac{f(x)}{x}\]
\begin{align*}
\lim_{x \to -\infty}\frac{f(x)}{x}=\lim_{x \to -\infty}\frac{(1+x)e^{-x-1}}{x}:
\begin{cases}
\lim_{x \to -\infty}\frac{x+1}{x}=1\\
\lim_{x \to -\infty}e^{-x-1}=+\infty
\end{cases}
\text{Par produit, } \lim_{x \to -\infty}\frac{f(x)}{x}=+\infty
\end{align*}
\begin{center}
\textcolor{green}{ \boxed{\lim_{x \to -\infty}\frac{f(x)}{x}=+\infty} \text{ Donc Cf admet une branche parabolique de direction (oy) au voisinage de }$-\infty$ }
\end{center}

\item \underline{\textcolor{green}{En $+\infty$}}

\[\text{Comme} \lim_{x \to +\infty}f(x)=-\infty, \text{calculons } \lim_{x \to +\infty}\frac{f(x)}{x}\]
\begin{align*}
\lim_{x \to +\infty}\frac{f(x)}{x}&=\lim_{x \to +\infty}\frac{\frac{x^{2}}{\ln(x+1)}}{x}\\
&=\lim_{x \to +\infty}\frac{x}{\frac{\ln(x+1)}{x+1}\times(x+1)}:
\begin{cases}
\lim_{x \to +\infty}\frac{x+1}{x}=1\\
\lim_{x \to +\infty}\frac{1}{\frac{\ln(x+1)}{x+1}}=+\infty
\end{cases}
\\
\text{Par produit, } \lim_{x \to +\infty}\frac{f(x)}{x}=+\infty
\end{align*}

\begin{center}
\textcolor{green}{ \boxed{\lim_{x \to +\infty}\frac{f(x)}{x}=+\infty} \text{ Donc (Cf) admet une branche parabolique de direction (oy) au voisinage de }$+\infty$ }
\end{center}
\end{itemize}
\end{enumerate}
\item
\begin{enumerate}
\item[a.] Etudions la continuité de $f$ en $-1$ et en $0$. $\textbf{0,5 pt}$
\begin{itemize}

\item \underline{\textcolor{green}{Continuité de f en -1}}

\[f \text{est continue en -1 ssi } \lim_{x \to -1^{-}}f(x)=\lim_{x \to -1^{+}}f(x)=f(1)\]
\begin{itemize}
\item \underline{\textcolor{green}{En $-1^{-}$}}
\[
\lim_{x \to -1^{-}}f(x)=\lim_{x \to -1^{-}}(1+x)e^{-x-1}:
\begin{cases}
\lim_{x \to -1^{-}}(1+x)=0\\
\lim_{x \to -1^{-}}e^{-x-1}=1
\end{cases}
\]

\[\text{Par produit, }\textcolor{green}{ \boxed{ \lim_{x \to -1^{-}}f(x)=0 } }\]

\item \underline{\textcolor{green}{En $-1^{+}$}}
\[
\lim_{x \to -1^{+}}f(x)=\lim_{x \to -1^{+}}\frac{x^{2}}{\ln(x+1)}:
\begin{cases}
\lim_{x \to -1^{+}}x^{2}=1\\
\lim_{x \to -1^{+}}\ln(x+1)=\ln(0^{+})=-\infty
\end{cases}
\]
\[\text{Par quotient, } \textcolor{green}{ \boxed{ \lim_{x \to -1^{+}}f(x)=0} } \]
\end{itemize}
\[\textcolor{green}{ \text{Comme } \lim_{x \to -1^{-}}f(x)=\lim_{x \to -1^{+}}f(x)=f(-1) \text{ donc f est continue en -1 } }\]
\item \underline{\textcolor{green}{Continuité  de f en 0}}

\[f \text{est continue en 0 ssi } \lim_{x \to 0}f(x)=f(0)\]

\begin{align*}
\lim_{x \to 0}f(x)&=\lim_{x \to 0}\frac{x^{2}}{\ln(x+1)}\\
&=\lim_{x \to 0}\frac{x}{\frac{\ln(x+1)}{x}}:
\begin{cases}
\lim_{x \to 0}x=0\\
\lim_{x \to 0}\frac{\ln(x+1)}{x}=\lim_{X \to 1}\frac{\ln(X)}{X-1}=1
\end{cases}
\end{align*}
\[\textcolor{green}{\text{Par produit, }  \boxed{ \lim_{x \to 0}f(x)=0}  \text{ Or, comme f(0)=0 donc f est continue en 0 } }\]
\end{itemize}
\item[b.] Etudions la dérivabilité de $f$ en $-1$ et en $0$ et interprétons graphiquement les résultats.$\textbf{1 pt}$
\begin{itemize}
\item \underline{\textcolor{green}{Dérivabilité de f en -1}}

\[f \text{est continue en -1 ssi } \lim_{x \to -1^{-}}\frac{f(x)-f(-1)}{x+1}=\lim_{x \to -1^{+}}\frac{f(x)-f(-1)}{x+1} \]
\begin{itemize}
\item \underline{\textcolor{green}{En $-1^{-}$}}

\begin{align*}
\lim_{x \to -1^{-}}\frac{f(x)-f(-1)}{x+1}&=\lim_{x \to -1^{-}}\frac{(x+1)e^{-x-1}}{x+1}\\
&=\lim_{x \to -1^{-}}e^{-x-1}=1
\end{align*}
\[\textcolor{green}{ \boxed{ \lim_{x \to -1^{-}} \frac{f(x)-f(-1)}{x+1}=1} \text{ donc f est dérivable à gauche de -1} } \]

\item \underline{\textcolor{green}{En $-1^{+}$}}
\begin{align*}
\lim_{x \to -1^{+}}\frac{f(x)-f(-1)}{x+1}&=\lim_{x \to -1^{+}}\frac{\frac{x^{2}}{\ln(x+1)}}{x+1}\\
&=\lim_{x \to -1^{+}}\frac{x^{2}}{(x+1)\ln(x+1)}:
\begin{cases}
\lim_{x \to -1^{+}}x^{2}=1\\
\lim_{x \to -1^{+}}(x+1)\ln(x+1)=0^{+}
\end{cases}
\end{align*}
\[\text{Par quotient:} \textcolor{green}{ \boxed{ \lim_{x \to -1^{+}} \frac{f(x)-f(-1)}{x+1}=+\infty} \text{ donc f n'est pas dérivable à droite de -1} } \]

\[\textbf{Conclusion:} \boxed{ \lim_{x \to -1^{-}} \frac{f(x)-f(-1)}{x+1} \neq \lim_{x \to -1^{+}} \frac{f(x)-f(-1)}{x+1} } \text{ donc f n'est pas dérivable en -1}  \]
\end{itemize}

\item \underline{\textcolor{green}{Dérivabilité de f en 0}}
\[f \text{est dérivable en 0 ssi } \lim_{x \to 0}\frac{f(x)-f(0)}{x}=a\in\mathbb{R}\]
\begin{align*}
\lim_{x \to 0}\frac{f(x)-f(0)}{x}&=\lim_{x \to 0}\frac{\frac{x^{2}}{\ln(x+1)}}{x}\\
&=\lim_{x \to 0}\frac{x}{\ln(x+1)}\\
&=\lim_{x \to 0}\frac{1}{\frac{\ln(x+1)}{x}}:
\begin{cases}
\lim_{x \to 0}\frac{\ln(x+1)}{x}=\lim_{X \to 1}\frac{\ln(X)}{X-1}=1
\end{cases}
\end{align*}

\[\text{Par quotient:} \textcolor{green}{ \boxed{ \lim_{x \to 0}\frac{f(x)-f(0)}{x}=1} \text{ donc f est dérivable à 0} } \]

\item \underline{\textcolor{green}{Interprétions graphiquement des résultats.}}

\begin{itemize}
\item \underline{\textcolor{green}{En $-1^{-}$}}
\[\text{comme } \lim_{x \to -1^{-}} \frac{f(x)-f(-1)}{x+1}=1 \text{ alors (Cf) admet une démi-tangente à gauche de (Cf)} \]
\[\text{d'équation } y=x+1\]
\item \underline{\textcolor{green}{En $-1^{+}$}}
\[\text{comme } \lim_{x \to -1^{+}} \frac{f(x)-f(-1)}{x+1}=+\infty \text{ alors (Cf) admet une démi-tangente à droite de (Cf)} \]
\[\text{orientée vers le bas} \]
\item \underline{\textcolor{green}{En $0$}}
\[\text{comme } \lim_{x \to 0} \frac{f(x)-f(0)}{x}=1 \text{ alors (Cf) admet une tangente au point 0 } \]
\[\text{d'équation y=x} \]
\end{itemize}

\end{itemize}
\end{enumerate}
\item
\begin{enumerate}
\item[a.] Montrons que pour tout $x \in ]-1, +\infty[$ et $x\neq 0$ on a $f'(x)=\frac{-xg(x)}{\ln^{2}(x+1)}$

Si $x \in ]-1, +\infty[, f(x)=\frac{x^{2}}{\ln(x+1)}$

$f'(x)=\frac{2x\ln(x+1)-\frac{x^{2}}{x+1}}{\ln^{2}(x+1)}=\frac{-x\left( 2\ln(x+1)-\frac{x}{x+1}\right) }{\ln^{2}(x+1)}$ or $g(x)=-2\ln(x+1)+\frac{x}{x+1}$

\[\text{D'où  }\textcolor{green}{ \boxed{ f'(x)=\frac{-xg(x)}{\ln^{2}(x+1)} } } \text{sur} ]-1, +\infty[\]

Calculons $f'(x)$ sur $]-\infty, -1[$. $\textbf{0,5 pt}$

Si $]-\infty, -1[, f(x)=(x+1)e^{-x-1}$

$ f'(x)=(x+1)e^{-x-1}=e^{-x-1}-(x+1)e^{-x-1}=-xe^{-x-1} $

\[\text{D'où  }\textcolor{green}{ \boxed{ f'(x)=-xe^{-x-1} } }\text{sur} ]-\infty, -1[\]

\item[b.] Etudions les variations de $f$ 
\begin{itemize}
\item Variation de f sur $]-1, +\infty[$:
\[\text{sur } ]-1, +\infty[,f'(x)=\frac{-xg(x)}{\ln^{2}(x+1)} \text{ le signe de f' dépend de son numérateur car } \]
\[\forall x \in ]-1, +\infty[, \ln^{2}(x+1)>0\]

\[\text{D'après la partie A question b), }\]

\[ \forall x \in \left( ]-1;\alpha[\right),\quad -x>0 \text{ et } g(x)<0 \text{ donc } -xg(x)<0\implies f'(x)<0\]
\[ \forall x \in \left( [0;+\infty[\right),\quad -x<0 \text{ et } g(x)<0 \text{ donc } -xg(x)\geq 0\implies f'(x)\geq 0\]
\[ \forall x \in \left( [\alpha;0]\right),\quad -x>0 \text{ et } g(x)>0 \text{ donc } -xg(x)\geq 0\implies f'(x)\geq 0\]

En résumé :
\textcolor{green}{
\[ \forall x \in \left( ]-1;\alpha[ \right), f'(x) < 0 \text{ donc f est décroissante }\]
\[ \forall x \in \left( [\alpha;+\infty[ \right)  , f'(x) \geq 0 \text{ donc f est décroissante      }\]
}
\item Variation de f sur $]-\infty, -1[$:
\[\text{sur } ]-\infty, -1[,f'(x)=-xe^{-x-1} \text{ le signe de f' dépend de -x } \]
\textcolor{green}{
\[\forall \in  ]-\infty, -1[, -x>0, f'(x)>0 \text{ donc f est croissante }\]
}
\end{itemize}
Dressons son tableau de variations.$\textbf{1 pt}$

\begin{tikzpicture}
 \tkzTabInit{$x$/1,$f'$/1,$f$/2}{$-\infty$,$-1$,$\alpha$,$+\infty$}
 \tkzTabLine{,+,z,-,z,+}
 \tkzTabVar{-/$-\infty$,+/$0$,-/$f(\alpha)$,+/$+\infty$}
 \tkzTabVal{2}{4}{0.8}{$0$}{$0$}
\end{tikzpicture}
\end{enumerate}
\item Soit $h$ la restriction de $f$ à $x \in ]0, +\infty[$.
\begin{enumerate}
\item[a.] Montrons que $h$ réalise une bijection de $]0, +\infty[$ sur un intervalle $J$ à préciser. $\textbf{0,25 pt}$

Sur $ ]0, +\infty[$, $f(x)=\frac{x^{2}}{\ln(x+1)}$ donc $h(x)=\frac{x^{2}}{\ln(x+1)}$

Comme $h$ est continue est croissante sur $]0, +\infty[$, donc il réalise une bijection de $]0, +\infty[$ vers $]0, +\infty[$.

D'où $J=]0, +\infty[$ 
\item[b.] Donnons le sens de variation de $h^{-1}$. $\textbf{0,25 pt}$

$h^{-1}$ a le même sens de variation que $h$ donc $h^{-1}$ est croissante sur $]0, +\infty[$.
\item[c.] Construisons $Cf$ et $Ch^{-1}$. $\textbf{1,25 pt}$
\begin{figure}[h]
\centering
\includegraphics[width=0.8\textwidth]{courbe2016.png}
\caption{Courbe de (Cf)}
\label{fig:monimage}
\end{figure}
\newpage
\begin{figure}[h]
\centering
\includegraphics[width=0.8\textwidth]{Courbe-2016.png}
\caption{Courbe de (Cf)}
\label{fig:monimage}
\end{figure}
\end{enumerate}
\end{enumerate}
%\subsection*{\centering Partie C}
%Soit m la fonction définie par $m(x)=\frac{\ln(x+1)}{x^{2}}-\frac{1}{x(x+1)}$
%\begin{enumerate}
%\item
%\begin{enumerate}
%\item[a.] Déterminer les fonctions $u$ et $v$ telles que pour tout $x \in ]0, +\infty[$ ,\\ $m(x) = u'(x)v(x)+u(x)v'(x)$. $\textbf{0,25 pt}$
%\item[b.] En déduire la fonction $H$ définie sur $]0, +\infty[$ telle que $H (x) = m(x)$ puis calculer 
%$\int_{1}^{2}\frac{1}{f(x)}dx$.$\textbf{ 0,75 pt}$
%\end{enumerate}
%\end{enumerate}
\end{document}
\documentclass[12pt]{article}
\usepackage{stmaryrd}
\usepackage{graphicx}
\usepackage[utf8]{inputenc}

\usepackage[french]{babel}
\usepackage[T1]{fontenc}
\usepackage{hyperref}
\usepackage{verbatim}

\usepackage{color, soul}

\usepackage{pgfplots}
\pgfplotsset{compat=1.15}
\usepackage{mathrsfs}

\usepackage{amsmath}
\usepackage{amsfonts}
\usepackage{amssymb}
\usepackage{tkz-tab}

\usepackage{tikz}
\usetikzlibrary{arrows, shapes.geometric, fit}


\usepackage[margin=2cm]{geometry}

\begin{document}

\begin{minipage}{0.8\textwidth}
	Pathé BA                          
\end{minipage}
\begin{minipage}{0.8\textwidth}
	BAC 2016
\end{minipage}

\begin{center}
\textbf{{\underline{\textcolor{red}{Premier Groupe Correction}}}}
\end{center}
\section*{\textcolor{green}{\underline{Correction Exercice 1} (6 points) :}}
\begin{enumerate}
	\item On considère l’équation $(E): z^{3}-13z^{2}+59z-87=0$ , où z est un nombre complexe.
	\begin{enumerate}
		\item Déterminons la solution réelle de $(E)$.\textbf{ 0,5pt}
		
		Si la solution est réelle alors elle est de la forme $z_{0}=a$ où $a\in \mathbb{R}$
		
		donc $a^{3}-13a^{2}+59a-87=0$ en cherchant parmis les solutions évidentes, on trouve que $a=3$
		
		\item Résolvons dans l’ensemble des nombres complexes $\mathbb{C}$ l’équation $(E)$.\textbf{ 0,5pt}
		
		Par Honer:\\
		\begin{tabular}{|c|c|c|c|c|}
					\hline
					& 1 & -13 & 59 & -87\\
					\hline
					3 &  & 3 & 30 & 87\\
					\hline
					& 1 & 10 & 29 & 0\\
					\hline
		\end{tabular}
		
		Donc $z^{3}-13z^{2}+59z-87=(z-3)(z^{2}+10z+29)=0$
		
		Résolvons $z^{2}+10z+29=0$
		
		$\Delta' = 25-29=-4\implies \Delta'=(2i)^{2}$ ou $\Delta'=(-2i)^{2}$
		
		$z_{1}=-5-2i$, $z_{2}=-5+2i$
		
		$z_{1}=-5-2i$, $z_{2}=-5+2i$
		
		\textcolor{green}{\boxed{S=\left\lbrace 3, -5-2i, -5+2i \right\rbrace }}
	\end{enumerate}
	\item On pose $a = 3$, $b = 5 - 2i$ et $c = 5 + 2i$.
	
			Le plan complexe étant muni d’un repère orthonormé direct $( O, \vec{u} ,\vec{v} )$, on considère les points A, B et C d’affixes respectives a, b et c. Soit M le point d’affixe z distinct de A et de B.
	\begin{enumerate}
		\item Calculons $\frac{b-a}{c-a}$.
		
		$\frac{b-a}{c-a}=\frac{2 - 2i}{2 + 2i}=\frac{(2 - 2i)(2 - 2i)}{4}=-2i$
		
		Déduisons la nature du triangle ABC. \textbf{ 0,5 + 0,5pt}
		
		$\arg\left( \frac{b-a}{c-a} \right) =-\frac{\pi}{2}[2\pi] $
		
		Comme $\arg\left( \frac{b-a}{c-a} \right) =(\overrightarrow{AC},\overrightarrow{AB})=-\frac{\pi}{2}[2\pi] $ et $\mid\frac{b-a}{c-a}\mid=2$ ( c'est-à-dire AB=2AC ) donc:
		
		ABC est un triangle rectangle en A.
		
		\begin{align*}\textbf{\textcolor{red}{Rappel:}}\\
				\arg\left( \frac{b-a}{c-a} \right) &=\arg(b-a)-\arg(c-a)\\
				&=(\vec{u},\overrightarrow{AB})-(\vec{u},\overrightarrow{AC})\\
				&=(\vec{u},\overrightarrow{AB})+(\overrightarrow{AC},\vec{u})\\
				&=(\overrightarrow{AC},\vec{u})+(\vec{u},\overrightarrow{AB})\\
				&=(\overrightarrow{AC},\overrightarrow{AB})
		\end{align*}

		
		\item On pose $Z=\frac{z-2}{z-5+2i}.$
		
		Donnons une interprétation géométrique de l’argument de Z.\textbf{ 0,5pt}
		
		$Z=\frac{z-2}{z-5+2i}=\frac{z-(2)}{z-(5-2i)}=\frac{z_{M}-z_{A}}{z_{M}-z_{B}}$
		
		$ \arg(Z)=\arg(\frac{z-z_{A}}{z-z_{B}})=(\overrightarrow{BM},\overrightarrow{AM})=-(\overrightarrow{AM},\overrightarrow{BM})=(\overrightarrow{MA},\overrightarrow{MB})$
		
		\textcolor{green}{L'argument de Z représente l'angle orienté entre les segments $\overrightarrow{BM}$ et $\overrightarrow{AM}$.}
		
		Déduisons-en l’ensemble des points M d’affixe z tels que Z soit un nombre réel non nul.\textbf{ 0,5pt}
		
		Si Z est un nombre réel non nul alors \\$ \arg(Z)=0\implies \arg(\frac{z-z_{A}}{z-z_{B}})=(\overrightarrow{BM},\overrightarrow{AM})=(\overrightarrow{MA},\overrightarrow{MB})=0$.
		
		 Donc l’ensemble des points $M$ tels que Z soit un nombre réel
  \end{enumerate}
%   \item Soit $(C)$ le cercle circonscrit au triangle ABC et $I$ le point d’affixe $2 - i$.\textbf{ 0,5pt}
%   \begin{enumerate}
%     \item Donner l’écriture complexe de la rotation $r$ de centre $I$ et d’angle $\frac{-\pi}{2}$.\textbf{ 0,5pt}
%     \item Déterminer l’image $(C")$ de $(C)$ par $r$. Construire $(C")$.\textbf{ 0,5pt}
%   \end{enumerate}
\end{enumerate}
\section*{\textcolor{green}{\underline{Correction Exercice 2} (6 points) :}}
À l’occasion de ses activités culturelles, le FOSCO d’un lycée organise un jeu pour le collectif des professeurs. Une urne contenant 4 boules rouges et une boule jaune indiscernables au toucher est placée dans la cour de l’école. Chaque professeur tire simultanément 2 boules de l’urne.

- Si les deux boules sont de même couleur, il les remet dans l’urne et procède à un second tirage successif avec remise de 2 autres boules.

- Si les deux boules sont de couleurs distinctes, il les remet toujours dans l’urne, mais dans ce cas le second tirage de 2 autres boules s’effectue successivement sans remise.

\textcolor{green}{1. Calculons la probabilité des événements suivants :}

\begin{itemize}
\item \underline{\textcolor{green}{Pour l'évènement A:}}

$P(A)=\frac{card(A)}{card(\Omega)}$

\textcolor{green}{Cherchons $card(\Omega)$ et card(A)}
	\begin{itemize}
	\item \underline{Pour $card(\Omega)$}
	\begin{align*}
	card(\Omega)&=C_{5}^{2}\\
			&=10
	\end{align*}
	
	\begin{center}
		\textcolor{blue}{\boxed{card(\Omega)=10}}
	\end{center}
	\item \underline{Pour $card(A)$}
	
		A : << Le professeur tire 2 boules de même couleur au premier tirage.>>

		\textcolor{green}{Autrement dit}: A : << RR >>. C'est-à-dire tirer 2 rouge parmis les 4 rouges.

		Donc card(A)=$C_{4}^{2}$

		\begin{align*}
			card(A)&=C_{4}^{2}\\
					&=6
		\end{align*}

		\begin{center}
			\textcolor{blue}{\boxed{card(A)=6}}
		\end{center}
	\end{itemize}
	Ainsi, la probabilité de l'évènement A est:

	$P(A)=\frac{card(A)}{card(\Omega)}=\frac{6}{10}$
		\begin{center}
			\textcolor{green}{\boxed{P(A)=\frac{3}{5}}}
		\end{center}
\item \underline{\textcolor{green}{Pour l'évènement B:}}

B : << Le professeur tire deux boules de couleurs différentes au premier tirage.>>  \textbf{0,25 pt}
			\textcolor{green}{Autrement dit}: B : << RJ >>. C'est-à-dire tirer 1 rouge parmis les 4 rouges et 1 jaune .

		\begin{align*}
			card(B)&=C_{4}^{1}\times C_{1}^{1}\\
					&=4
		\end{align*}

		\begin{center}
			\textcolor{blue}{\boxed{card(B)=4}}
		\end{center}

	Ainsi, la probabilité de l'évènement B est:

	$P(B)=\frac{card(B)}{card(\Omega)}=\frac{4}{10}$
		\begin{center}
			\textcolor{green}{\boxed{P(B)=\frac{2}{5}}}
		\end{center}
\end{itemize}

\textcolor{green}{En remarquant que les évènements A et B sont des évènements contraires, et que les évènements C, D, E et F sont dépendants soit de la réalisation de A ou de B, contruisons un arbe de probabilité.}

\begin{tikzpicture}[level distance=3cm,
  level 1/.style={sibling distance=6cm},%Ecarte les branches des 1eme ramifications
  level 2/.style={sibling distance=2.5cm},%Ecarte les branches des  2eme ramifications
  %level 3/.style={sibling distance=2cm}]%Ecarte les branches des 3eme ramifications
    every node/.style={text width=2cm, align=center}]%Permet de spécifier une largeur pour chaque nœud
  \node {}
    child {node {$B$}
      child {node {$RR$}    
      }
      child {node {$JR$}    
      }
    }      
    child {node {$A$} 
      child {node {$JR$}    
      }
      child {node {$JJ$}    
      } 
      child {node {$RR$}    
      }
    };
\node at (-4,-1.5) [right] {$\frac{2}{5}$};
\node at (1.3,-1.5) [right] {$\frac{3}{5}$};

\node at (-5,-4) [right] {$\frac{12}{20}$};
\node at (-3,-4) [right] {$\frac{8}{20}$};

\node at (-0.1,-4) [right] {$\frac{8}{25}$};
\node at (1.5,-5.2) [right] {$\frac{1}{25}$};
\node at (3.9,-4.5) [right] {$\frac{16}{25}$};

\end{tikzpicture}

\begin{itemize}
\item \underline{\textcolor{green}{Pour P(C):Tirage successif avec remise de 2 autres boules}}

C : << Le professeur tire deux boules de même couleur au second tirage sachant que les boules tirées au premier tirage sont de même couleur.>>  \textbf{0,5 pt}

C : << RR ou JJ >>

P(C)=P(RR ou JJ)=$P_{A}({RR-ou-JJ})=P(RR)+P(JJ)=\frac{16}{25}+\frac{1}{25}=\frac{17}{25}$

		\begin{center}
			\textcolor{green}{\boxed{P(C)=\frac{17}{25}}}
		\end{center}

\item \underline{\textcolor{green}{Pour P(D):tirage successif sans remise de 2 autres boules}}

D : << Le professeur tire deux boules de même couleur au second tirage sachant que les boules tirées au premier tirage sont de couleurs distinctes.>> \textbf{0,5 pt}

D : << RJ-JR>> 

\begin{tikzpicture}
  % Case 1
  \draw (0,0) rectangle (1.5,1.5);
  \node at (0.5,1.9) {R};
  \node at (0.8,0.8) {4};
  \node at (3,0.8) {X};
  
  % Case 2
  \draw (5,0) rectangle (6.5,1.5);
  \node at (5.7,1.8) {J};
  \node at (5.7,0.8) {1};
  \node at (8,0.8) {X};
  \node at (9,0.8) {2};
\end{tikzpicture}\\ 

P(D)=P(RJ)=$P_{B}({RJ})=P(RJ)=\frac{4\times 2}{20}=\frac{2}{5}$

		\begin{center}
			\textcolor{green}{\boxed{P(D)=\frac{2}{5}}}
		\end{center}

\item \underline{\textcolor{green}{Pour P(E):}}

E : << Le professeur tire 2 boules de couleurs distinctes au second tirage sachant que les boules tirées au premier tirage sont de couleurs distinctes.>> \textbf{0,5 pt}

E : << RR >>

P(E)=P(RR)=$P_{B}({RR})=P(RR)=\frac{12}{20}=\frac{3}{5}$

		\begin{center}
			\textcolor{green}{\boxed{P(D)=\frac{3}{5}}}
		\end{center}

\item \underline{\textcolor{green}{Pour P(F):}}

F : << Le professeur tire 2 boules de couleurs distinctes au premier et au second tirage. >> \textbf{0,5 pt}

E : << JR >>

P(F)=P(JR)=$P(A\cap F)=\frac{2}{5}\times \frac{2}{5}$

		\begin{center}
			\textcolor{green}{\boxed{P(A\cap F)=\frac{4}{25}}}
		\end{center}
\end{itemize}
2. Pour le second tirage, chaque boule rouge tirée fait gagner au FOSCO 1000 F et chaque boule jaune tirée fait gagner au collectif des professeurs 1000 F.
Soit \(X\) la variable aléatoire à laquelle on associe le gain obtenu par le FOSCO.

a. Déterminer les différentes valeurs prises par \(X\) et sa loi de probabilité. \textbf{1 pt}

b. Déterminer la fonction de répartition de \(X\). \textbf{1 pt}

3. Étant donné que le collectif est composé de 50 professeurs qui ont tous joué indépendamment et dans les mêmes conditions, déterminer la probabilité des événements suivants :

G : << le FOSCO réalise un gain de 100 000 F. >> \textbf{0,5 pt}

H : << le collectif des professeurs réalise un gain de 100 000 F. >> \textbf{0,5 pt}

I : << Ni gagnant, ni perdant. >> \textbf{0,5 pt}

\end{document}
